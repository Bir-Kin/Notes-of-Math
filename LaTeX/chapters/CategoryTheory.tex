\chapter{定向极限的对偶和对偶的逆向极限}
\begin{definition}[定向极限]
    设 $I$ 是一个定向集, $\{V_i\}_{i\in I}$ 是一族以 
    $I$ 为指标集的域 $\mathbb{F}$ 上的线性空间, 
    满足对任意 $i\ls j$ 都有线性映射 
    $\varphi^i_j:V_i\to V_j$. 
    基于此, 我们定义定向极限
    \begin{equation*}
        \varinjlim_{i\in I} V_i := 
        \bigg(\bigoplus_{i\in I} V_i\bigg)/\sim
    \end{equation*}
    其中等价关系 $\sim$ 定义为:
    \begin{equation*}
        v_i \sim v_j \iff \exists\, k\gs i,j, 
        \text{ s.t. }\varphi^i_k(v_i)=\varphi^j_k(v_j).
    \end{equation*}
\end{definition}

\begin{definition}[逆向极限]
    条件同上, 对于对偶空间 $V_i^*:=\Hom(V_i,\mbb{F})$ 
    当 $i\ls j$ 时, 存在 $(\varphi^i_j)^*:V_j^*\ra V_i^*$. 
    基于此我们定义逆向极限
    \begin{equation*}
        \varprojlim_{i\in I} V_i^* := 
        \Big\{\{f_i\}_{i\in I}\in\prod_{i \in I}V_i^*\,\Big|\,
        f_i=f_j\circ\varphi^i_j,\text{ for }\forall\, i\ls j\Big\}
    \end{equation*}
\end{definition}

从定义能看出对偶的关系, 定向极限是直和对象的商对象, 
逆向极限是直积对象的子对象. 事实上它们确实存在自然的同构关系:

\begin{theorem}[定向极限的对偶是对偶的逆向极限]
    \begin{equation*}
        \Hom\left(\varinjlim_{i\in I}V_i,\mathbb{F}\right)
        \cong\varprojlim_{i\in I}\,\Hom(V_i,\mathbb{F}).
    \end{equation*}
    也即
    \begin{equation*}
        \left(\varinjlim_{i\in I}V_i\right)^*
        \cong\varprojlim_{i\in I}V_i^*.
    \end{equation*}
\end{theorem}
\begin{proof}
    我们分别定义 
    \begin{align*}
        \Phi&:\left(\varinjlim V_i\right)^*
        \ra\varprojlim V_i^*, \\
        \Psi&:\varprojlim V_i^*\ra
        \left(\varinjlim V_i\right)^*.
    \end{align*}
    并说明它们互为对方的逆映射.

    $1^{\circ}$ 定义 $\Phi$: 对任意 
    $f:\varinjlim V_i\ra\mbb{F}$, 对每一个 $i\in I$, 
    令 
    \begin{equation*}
        f_i:=f\circ \pi_i:V_i\ra\varinjlim V_i\ra\mbb{F}
    \end{equation*}
    则当 $j\ls k$ 时, 有如下交换图
    \begin{center}
        \begin{tikzcd}
            V_j \arrow[r, "\pi_j"] \arrow[d, "\varphi^j_k"'] & \varinjlim V_i \\
            V_k \arrow[ru, "\pi_k"']                         &               
        \end{tikzcd}
    \end{center}
    因此
    \begin{equation*}
        f_j=f\circ\pi_j
        =f\circ\pi_k\circ\varphi^j_k=f_k\circ\varphi^j_k
    \end{equation*}
    这说明 $\{f_i\}_{i\in I}$ 定义了 $\varprojlim V^*_i$ 
    中的一个元素. 
    令 
    \begin{equation*}
        \Phi(f):=\{f_i\}_{i\in I}=\{f\circ\pi_i\}_{i\in I}.
    \end{equation*}

    $2^{\circ}$ 定义 $\Psi$: 对任意 
    $\{g_i\}_{i\in I}\in\varprojlim V^*_i$, 定义 
    \begin{equation*}
        \Psi(\{g_i\})([v_j]):=g_j(v_j),\quad
        \forall\,[v_j]\in\varinjlim V_i.
    \end{equation*}
    为了验证其良定义性, 注意到逆向极限保证了对于任意 $j\ls k$, 
    都有 
    \begin{equation*}
        g_j = g_k \circ \varphi^j_k.
    \end{equation*}
    因此当 $[v_j]=[v_k]$ 时, 
    由定向极限的定义知存在 $l\gs j,k$ 使得 
    \begin{equation*}
        \varphi^j_l(v_j)=\varphi^k_l(v_k)
    \end{equation*} 
    因此 
    \begin{equation*}
        g_j(v_j) = g_l(\varphi^j_l(v_j))
        =g_l(\varphi^k_l(v_k))=g_k(v_k).
    \end{equation*}
    这说明我们的定义是良好的. 

    从定义不难看出对任意 $\{g_i\}_{i\in I}\in\varprojlim V^*_i$,
    \begin{equation*}
        \Psi(\{g_i\})\circ\pi_j=g_j,\quad\forall\,j\in I.
    \end{equation*}
    因此 $\Phi\circ\Psi=\id$. 另一方面对任意 
    $f:\varinjlim V_i\ra\mbb{F}$, 
    \begin{equation*}
        \Psi(\{f_i\})([v_j])=f\circ\pi_j(v_j)=f([v_j]).
    \end{equation*}
    因此 $\Psi\circ\Phi=\id$.
\end{proof}