\chapter{复几何、多复变}
    \section{实线性空间与复线性空间}
        流形 $\mathbb{C}^n$ 中每点的坐标为 $(z^1,\dots,z^n)$, 记 $z^j=x^j+iy^j$, 于是 $x^1,\dots,x^n$ 和 $y^1,\dots,y^n$ 是 $\mathbb{C}^n$ 的实坐标.
        对 $\forall p\in\mathbb{C}^n$, 切空间 $T_p\mathbb{C}^n$ 有一组基 $\pd{}{x^1},\dots,\pd{}{x^n}$, $\pd{}{y^1},\dots,\pd{}{y^n}$; 余切空间 $T^*_p\mathbb{C}$
        有一组基 $\md x^1,\dots,\md x^n$, $\md y^1,\dots,\md y^n$.

        定义 $J_p:T_p\mathbb{C}^n\rightarrow T_p\mathbb{C}^n$
        \begin{equation*}
            J_p\left(\pd{}{x^j}\right)=\pd{}{y^j},\qquad J_p\left(\pd{}{y^j}\right)=-\pd{}{x^j}
        \end{equation*}
        它的对偶变换为 $J^*_p:T^*_p\mathbb{C}^n\rightarrow T^*_p\mathbb{C}^n$
        \begin{equation*}
            J^*_p\left(\md x^j\right)=-\md{y^j},\qquad J^*_p\left(\md{y^j}\right)=\md{x^j}
        \end{equation*}
        因为 $(J_p)^2=-\mathrm{id}$, 所以 $(J^*_p)^2=-\mathrm{id}$, 于是 $J_p$ 和 $J^*_p$ 均可对角化且特征值均为 $\pm i$.

        计算可发现 $J^*_p$ 的属于 $i$ 的特征子空间的一组基为 $\md{z^1},\dots,\md{z^n}$, 属于 $-i$ 的特征子空间的一组基为 $\md{\bar{z}^1}\dots,\md{\bar{z}^n}$, 其中
        \begin{equation*}
            \md{z^j}=\md{x^j}+i\md{y^j},\qquad \md{\bar{z}^j}=\md{x^j}-i\md{y^j}
        \end{equation*}
        于是对应地它的对偶空间 $T_p\mathbb{C}^n\otimes\mathbb{C}$ 的属于 $i$ 的特征子空间的一组基为 $\pd{}{z^1},\dots\pd{}{z^n}$, 属于 $-i$ 的特征子空间的一组基为 $\pd{}{\bar{z}^1},\dots,\pd{}{\bar{z}^n}$, 其中
        \begin{equation}\label{partialz&partialbarz}
            \pd{}{z^j}=\frac{1}{2}\left(\pd{}{x^j}-i\pd{}{y^j}\right),\qquad \pd{}{\bar{z}^j}=\frac{1}{2}\left(\pd{}{x^j}+i\pd{}{y^j}\right)
        \end{equation}
        
        回顾 Cauchy-Riemann 方程: 设 $f=u+iv$ 是全纯函数, 则 $u(x,y),v(x,y)$ 关于 $x,y$ 的偏导数满足:
        \begin{equation*}
            \pd{u}{x}=\pd{v}{y},\qquad \pd{u}{y}=-\pd{v}{x}
        \end{equation*}
        代入 $\pd{f}{\bar{z}}$ 可得
        \begin{equation*}
            \pd{f}{\bar{z}}=\frac{1}{2}\left(\pd{f}{x}+i\pd{f}{y}\right)=\frac{1}{2}\left(\pd{u}{x}-\pd{v}{y}\right)+i\frac{1}{2}\left(\pd{v}{x}+i\pd{u}{y}\right)=0
        \end{equation*}
        于是 $f$ 全纯当且仅当 $\pd{f}{\bar{z}}\equiv0$, 推广到多元复变量即: $f(z^1,\dots,z^n)$ 是多元全纯函数当且仅当每个 $\pd{f}{\bar{z}^j}\equiv0$.
    
    \section{实可微与复可微}
        \subsection{一维情形: 复导数是复数}
            我们知道一个复变量复值(连续)函数 $f(z)$ 可以视作一个实二元向量值(连续)函数, 即有 $1-1$ 对应 $\mathcal{C}(\mC,\mC)\leftrightarrow\mathcal{C}(\mR^2,\mR^2)$:
            \begin{equation*}
                f(z) = u(x+\mi y)+\mi v(x+\mi y) \leftrightarrow \begin{pmatrix} u(x,y) \\ v(x,y) \end{pmatrix}.
            \end{equation*}
            回顾向量值函数可微性的定义, 如果
            \begin{equation}\label{R-differential}
                \begin{pmatrix}
                    u(x+\Delta x,y+\Delta y) \\ v(x+\Delta x,y+\Delta y)
                \end{pmatrix} - 
                \begin{pmatrix}
                    u(x,y) \\ v(x,y)
                \end{pmatrix} = 
                \begin{pmatrix}
                    a & b \\
                    c & d
                \end{pmatrix}\cdot
                \begin{pmatrix}
                    \Delta x \\ \Delta y 
                \end{pmatrix}+
                \begin{pmatrix}
                    o(\sqrt{(\Delta x)^2+(\Delta y)^2}) \\ o(\sqrt{(\Delta x)^2+(\Delta y)^2})
                \end{pmatrix},
            \end{equation}
            则称 $(u,v)'$ 在点 $(x,y)'$ 处可微, 
            此时 $\begin{pmatrix} a & b \\ c & d \end{pmatrix} = \begin{pmatrix} \pd{u}{x} & \pd{u}{y} \\ \pd{v}{x} & \pd{v}{y} \end{pmatrix}=:\frac{{\rm D}(u,v)}{{\rm D}(x,y)}$.

            复变量函数可微性定义如下, 若
            \begin{equation*}\label{C-differential}
                f(z+\Delta z)-f(z) = w\cdot\Delta z+o(\Delta z),
            \end{equation*}
            其中 $w=p+\mi q\in\mC$, 则称 $f$ 在点 $z$ 处复可微.
            
            这里为什么要强调 $w$ 是一个复数呢? 这是为了提醒我们可以将Jacobi矩阵 $
            \begin{pmatrix}
                a & b \\
                c & d
            \end{pmatrix}$ 看成一个复数, 那什么时候能把一个实 $2\times2$ 矩阵看成一个复数呢?

            事实上我们能很自然地给一个 $\mC$ 到 $\mR^{2\times2}$ 的嵌入:
            \begin{equation*}
                \begin{aligned}
                    \iota:\mC &   \hookrightarrow\mR^{2\times2} \\
                    z = a+\mi b &   \mapsto
                    \begin{pmatrix}
                        a & -b \\
                        b & a
                    \end{pmatrix}.
                \end{aligned}
            \end{equation*}
            \begin{remark}
                该嵌入是通过将数乘一个复数看成 $\mC(\cong\mR^2)$ 到自身的线性变换, 再取一组实线性无关的基 $1$、$\mi$, 
                这个线性变换在这组基下的矩阵就是映射的像.
            \end{remark}
            因此能将 $\pD{u,v}{x,y}$ 视作一个复数当且仅当存在 $p,q\in\mC$ 使得
            \begin{equation*}
                \begin{pmatrix} \pd{u}{x} & \pd{u}{y} \\ \pd{v}{x} & \pd{v}{y} \end{pmatrix} = 
                \begin{pmatrix} p & -q \\ q & p \end{pmatrix}
            \end{equation*}
            从而
            \begin{equation*}
                \left\{
                    \begin{aligned}
                        \pd{u}{x} &= \pd{v}{y} = p \\
                        \pd{v}{x} &= -\pd{u}{y} = q
                    \end{aligned}
                \right.
            \end{equation*}
            此即 Cauchy-Riemann 方程.

        \subsection{高维情形: 复微分是复线性变换}
            一维情形下复线性变换只有数乘, 这种特殊性会让我们错失更一般的图景. 
            设 $f(\ft{z}{1}{n}) = (f_1(\ft{z}{1}{n}),\dots,f_n(\ft{z}{1}{n}))'$ 为 $\mC^n$ 上的(连续)函数. 
            设 $f_i = u_i+\mi v_i,\,z_j = x_j+\mi y_j$, 则 $f$ 和 $\mR^{2n}$ 到自身的(连续)映射一一对应:
            \begin{equation*}
                f(\ft{z}{1}{n})\leftrightarrow
                \begin{pmatrix}
                    u_1(\cft{x}{1}{n},\cft{y}{1}{n}) \\
                    \vdots \\
                    u_n(\cft{x}{1}{n},\cft{y}{1}{n}) \\
                    v_1(\cft{x}{1}{n},\cft{y}{1}{n}) \\
                    \vdots \\
                    v_n(\cft{x}{1}{n},\cft{y}{1}{n}) \\
                \end{pmatrix}
            \end{equation*}
            当这个映射可微时, 我们也有它的 Jacobi 矩阵:
            \begin{equation*}
                {\rm D}f := 
                \begin{pmatrix}
                    \pD{\cft{u}{1}{n}}{\cft{x}{1}{n}} & \pD{\cft{u}{1}{n}}{\cft{y}{1}{n}} \\ \\
                    \pD{\cft{v}{1}{n}}{\cft{x}{1}{n}} & \pD{\cft{v}{1}{n}}{\cft{y}{1}{n}}
                \end{pmatrix} = 
                \begin{pmatrix}
                    \pd{u_1}{x_1}   &   \cdots  &   \pd{u_1}{x_n}   &   \pd{u_1}{y_1}   &   \cdots  &   \pd{u_1}{y_n}   \\
                    \vdots          &           &   \vdots          &   \vdots          &           &   \vdots          \\
                    \pd{u_n}{x_1}   &   \cdots  &   \pd{u_n}{x_n}   &   \pd{u_n}{y_1}   &   \cdots  &   \pd{u_n}{y_n}   \\
                    \pd{v_1}{x_1}   &   \cdots  &   \pd{v_1}{x_n}   &   \pd{v_1}{y_1}   &   \cdots  &   \pd{v_1}{y_n}   \\
                    \vdots          &           &   \vdots          &   \vdots          &           &   \vdots          \\
                    \pd{v_n}{x_1}   &   \cdots  &   \pd{v_n}{x_n}   &   \pd{v_n}{y_1}   &   \cdots  &   \pd{v_n}{y_n}   \\
                \end{pmatrix}
            \end{equation*}
            $Df$ 是一个实线性变换, 我们将说明当 $f$ 复可微(等价于满足高维C-R方程)的时候 $Df$ 是复线性的.

            我们先做一些线性代数的准备, $\mC^n$ 可看成 $\mR^{2n}$, 因此复线性变换能对应一个实线性变换:
            \begin{equation}\label{CnintoR2n}
                \begin{aligned}
                    \iota:\mC^{n\times n}   &   \hookrightarrow\mR^{2n\times 2n} \\
                    A + \mi B               &   \mapsto \begin{pmatrix} A & -B \\ B & A \end{pmatrix}
                \end{aligned}
            \end{equation}
            特别地, 数乘一个复数 $w=p+\mi q$ (记为 $\lambda_w$) 对应的矩阵为
            \begin{equation*}
                \lambda_{w}\mapsto \begin{pmatrix} pI_n & -qI_n \\ qI_n & pI_n \end{pmatrix} =
                p\begin{pmatrix} I_n & \\ & I_n \end{pmatrix} 
                + q{\setlength{\arraycolsep}{0.8pt} \begin{pmatrix} & -I_n \\ I_n & \end{pmatrix}} =: 
                pI_{2n} + qJ
            \end{equation*}

            反过来, 一个 $\mR^{2n}$ 上的线性变换 $\varphi$ 什么时候是复线性的呢? 将复线性的定义转化为实的语言即为:
            \begin{equation*}
                \varphi(\lambda_{w}(v)) = \lambda_{w}(\varphi(v)),\quad\forall\,v\in\mR^{2n},\,w\in\mC.
            \end{equation*}
            由 $\varphi$ 实线性, 实际上只需验证:
            \begin{equation*}
                \varphi(\lambda_{\mi}(v)) = \lambda_{\mi}(\varphi(v)),\quad\forall\,v\in\mR^{2n}.
            \end{equation*}
            写成矩阵形式即要求 $\varphi$ 对应的矩阵 $M_{\varphi}$ 满足:
            \begin{equation}\label{C-linear}
                M_{\varphi}\circ J = J\circ M_{\varphi}.
            \end{equation}
            设 $M_{\varphi} = \begin{pmatrix} A & C \\ B & D \end{pmatrix}$, 其中 $A,B,C,D\in\mR^{n\times n}$ 则
            \begin{equation*}
                \begin{pmatrix} A & C \\ B & D \end{pmatrix}\begin{pmatrix} & -I_n \\ I_n & \end{pmatrix} = 
                \begin{pmatrix} & -I_n \\ I_n & \end{pmatrix}\begin{pmatrix} A & C \\ B & D \end{pmatrix} \Rightarrow
                \left\{\begin{aligned} A &= D \\ B &= -C \end{aligned}\right. .
            \end{equation*}

            \begin{theorem}[复可微的含义]
                设 $f$ 是 $\mC^n$ 到 $\mC^n$ 的连续映射, $f$ 作为 $\mR^{2n}$ 到 $\mR^{2n}$ 的映射是光滑的, 则 $f$ 复线性当且仅当其微分
                ${\rm D}f$ 是复线性的.
            \end{theorem}
            \begin{proof}
                由前文的讨论知 ${\rm D}f$ 复线性当且仅当它满足
                \begin{equation*}
                    {\rm D}f\circ J = J\circ{\rm D}f\Leftrightarrow
                    \left\{
                        \begin{aligned}
                            \pD{\cft{u}{1}{n}}{\cft{x}{1}{n}} &= \pD{\cft{v}{1}{n}}{\cft{y}{1}{n}} \\
                            \pD{\cft{v}{1}{n}}{\cft{x}{1}{n}} &= -\pD{\cft{u}{1}{n}}{\cft{y}{1}{n}}
                        \end{aligned}
                    \right.\Leftrightarrow{\rm D}f\text{满足C-R方程}
                \end{equation*}
            \end{proof}
            \begin{remark}
                由此可以看出复函数的全纯性确实比光滑性更强, 它要求函数的实 {\rm Jacobian} 有对称性 {\rm(\ref{C-linear})}.
            \end{remark}
        
        \subsection{复导数存在与复线性}
            回到 $1$ 维情形, 复函数 $w = f(z)$ 在点 $z$ 处的定义为
            \begin{equation}\label{ComplexDerivative}
                f'(z) := \lim_{\Delta z\rightarrow0}\frac{f(z+\Delta z) - f(z)}{\Delta z}
            \end{equation}
            现假设 $f$ 光滑(不一定全纯), 则在点 $z=x+\mi y$ 处有 (\ref{R-differential}) 成立, 
            观察到在 (\ref{R-differential}) 两侧乘以行向量 
            {\setlength{\arraycolsep}{0.8pt} $\begin{pmatrix} 1 & \mi \end{pmatrix}$} 能得到
            \begin{equation*}
                \Delta f = \Delta u+\mi\Delta v = {\setlength{\arraycolsep}{0.8pt}\begin{pmatrix}  1 & \mi \end{pmatrix}}
                \pD{u,v}{x,y}
                \begin{pmatrix} \Delta x \\ \Delta y \end{pmatrix} + o\left(\sqrt{(\Delta x)^2+(\Delta y)^2}\right)
            \end{equation*}
            为了配出导数定义式 (\ref{ComplexDerivative}) 的除法我们需要存一个复数 $w=p+\mi q$ 使得
            \begin{align*}
                {\setlength{\arraycolsep}{0.8pt}\begin{pmatrix} 1 & \mi \end{pmatrix}}
                \pD{u,v}{x,y}
                \begin{pmatrix} \Delta x \\ \Delta y \end{pmatrix} &= 
                w\cdot\Delta z = (p+\mi q)(\Delta x+\mi\Delta y) \\
                &={\setlength{\arraycolsep}{0.8pt}\begin{pmatrix} 1 & \mi \end{pmatrix}}
                \begin{pmatrix} p \\ q \end{pmatrix}
                {\setlength{\arraycolsep}{0.8pt}\begin{pmatrix} 1 & \mi \end{pmatrix}}
                \begin{pmatrix} \Delta x \\ \Delta y \end{pmatrix}
            \end{align*}
            因此 
            \begin{equation*}
                {\setlength{\arraycolsep}{0.8pt}\begin{pmatrix} 1 & \mi \end{pmatrix}}
                \pD{u,v}{x,y} = 
                {\setlength{\arraycolsep}{0.8pt}\begin{pmatrix} 1 & \mi \end{pmatrix}}
                \begin{pmatrix} p \\ q \end{pmatrix}
                {\setlength{\arraycolsep}{0.8pt}\begin{pmatrix} 1 & \mi \end{pmatrix}}
            \end{equation*}
            因为此时 
            \begin{equation*}
                {\setlength{\arraycolsep}{0.8pt}\begin{pmatrix} 1 & \mi \end{pmatrix}J 
                = \begin{pmatrix} 1 & \mi \end{pmatrix}\begin{pmatrix} & -1 \\ 1 & \end{pmatrix} 
                = \begin{pmatrix} \mi & -1 \end{pmatrix} = \mi\begin{pmatrix} 1 & \mi \end{pmatrix}}
            \end{equation*}
            所以
            \begin{align*}
                {\setlength{\arraycolsep}{0.8pt}\begin{pmatrix} 1 & \mi \end{pmatrix}}\pD{u,v}{x,y}J 
                &= {\setlength{\arraycolsep}{0.8pt}\begin{pmatrix} 1 & \mi \end{pmatrix}}
                \begin{pmatrix} p \\ q \end{pmatrix}
                {\setlength{\arraycolsep}{0.8pt}\begin{pmatrix} 1 & \mi \end{pmatrix}}J \\
                &= \mi\cdot{\setlength{\arraycolsep}{0.8pt}\begin{pmatrix} 1 & \mi \end{pmatrix}}
                \begin{pmatrix} p \\ q \end{pmatrix}
                {\setlength{\arraycolsep}{0.8pt}\begin{pmatrix} 1 & \mi \end{pmatrix}} \\
                &= \mi\cdot{\setlength{\arraycolsep}{0.8pt}\begin{pmatrix} 1 & \mi \end{pmatrix}}\pD{u,v}{x,y}J \\
                &= {\setlength{\arraycolsep}{0.8pt}\begin{pmatrix} 1 & \mi \end{pmatrix}}J\pD{u,v}{x,y}
            \end{align*}
            所以 Jacobian 需要满足
            \begin{equation*}
                \pD{u,v}{x,y}J = J\pD{u,v}{x,y}
            \end{equation*}
            这就与前面的复线性联系上了.
            \begin{remark}
                或许可以直接用
                \begin{equation*}
                    {\setlength{\arraycolsep}{0.8pt}\begin{pmatrix} 1 & \mi \end{pmatrix}}
                    \pD{u,v}{x,y} = 
                    {\setlength{\arraycolsep}{0.8pt}\begin{pmatrix} 1 & \mi \end{pmatrix}}
                    \begin{pmatrix} p \\ q \end{pmatrix}
                    {\setlength{\arraycolsep}{0.8pt}\begin{pmatrix} 1 & \mi \end{pmatrix}} = 
                    (p+\mi q)\cdot{\setlength{\arraycolsep}{0.8pt}\begin{pmatrix} 1 & \mi \end{pmatrix}}
                \end{equation*}
                说明 $\pD{u,v}{x,y}$ 是复线性的?
            \end{remark}
        
        \subsection{全纯部分与反全纯部分} 
            前面我们考虑了 $\mC^n$ 到 $\mR^{2n}$ 的嵌入 $\iota$ (详见(\ref{CnintoR2n})), 实 $2n$ 阶矩阵 $M\in{\rm Im}\,\iota$ 当且仅当 $MJ=JM$.
            现在若 $MJ=-JM$, 可以算出 $M$ 形如 $\begin{pmatrix} A & B \\ B & -A \end{pmatrix}$, 定义
            \begin{equation}\label{CnintoR2n-anti}
                \begin{aligned}
                    \bar{\iota}:\mC^{n\times n} &   \hookrightarrow\mR^{2n\times2n} \\
                    A + \mi B                   &   \mapsto \begin{pmatrix} A & B \\ B & -A \end{pmatrix}
                \end{aligned}
            \end{equation}
            \begin{proposition}
                $\mR^{2n\times2n} = {\rm Im}\,\iota\oplus{\rm Im}\,\bar{\iota}$, 分别记 $\mR^{2n\times2n}$ 到两个分量的投影为 $\partial$、$\bar{\partial}$, 
                (我想把它们分别称为全纯部分和反全纯部分)
            \end{proposition}
            \begin{proof}
                ${\rm Im}\,\iota\cap{\rm Im}\,\bar{\iota}=\emptyset$ 显然, 仅需证 $\mR^{2n\times2n} = {\rm Im}\,\iota+{\rm Im}\,\bar{\iota}$,
                当然可以待定系数
                \begin{equation*}
                    M =  \begin{pmatrix} M_{11} & M_{12} \\ M_{21} & M_{22} \end{pmatrix} = 
                    \begin{pmatrix} A_1 & -B_1 \\ B_1 & A_1 \end{pmatrix} + 
                    \begin{pmatrix} A_2 & B_2 \\ B_2 & -A_2 \end{pmatrix}
                \end{equation*}
                解出 $A_1,\,B_1,\,A_2,\,B_2$. 下面用 $J$ 给出一个具体表达式, 设 $M = \partial M + \bar{\partial}M$, 则
                \begin{equation*}
                    JMJ = J\partial MJ + J\bar{\partial}MJ = -\partial M + \bar{\partial}M
                \end{equation*}
                因此
                \begin{equation}
                    \begin{aligned}
                        \partial M &= \frac{1}{2}(M - JMJ) \\
                    \bar{\partial}M &= \frac{1}{2}(M + JMJ)
                    \end{aligned}
                \end{equation}
            \end{proof}
            \begin{remark}
                最后算出来的表达式和 {\rm (\ref{partialz&partialbarz})} 很像.
            \end{remark}
            \begin{remark}
                一般复光滑函数的实{\rm Jacobian}也可以分解成这两部分, 这与复几何的 $T_{\mR}X\otimes\mC = T^{1,0}X\oplus T^{0,1}X$ 有没有联系?
            \end{remark}
    
    \section{复流形的例子}
        \begin{example}[复射影平面是黎曼球面]
            黎曼球面是 $\mR^3$ 中的单位球面, 它有如下整体坐标: 
            \begin{equation*}
                S^2 = \left\{(x,y,z)\in\mR^3\,\big|\,x^2+y^2+z^2=1\right\}.
            \end{equation*}
            复射影平面 $\mCP$ 则为:
            \begin{equation*}
                \mCP = \left\{[u:v]\,\big|\,u,v\text{不同时为零}\right\}.
            \end{equation*}
            我们知道 $S^2$ 有一个常用的坐标覆盖: $S^2\backslash\{N\},\,S^2\backslash\{S\}$, $\mCP$ 有一个常用的坐标覆盖 $U_0,\,U_1$. 
            可分块定义同构映射:
            \begin{center}
                \begin{tikzcd}[column sep=large,row sep=tiny]
                    {} & {S^2\backslash\{N\}} & {U_0} & {} \\
                    {} & {(x,y,z)} & {\left[1:\frac{x+\sqrt{-1}y}{1-z}\right]} & {} \\
                    & {\left(\frac{2{\rm Re}w_0}{|w_0|^2+1},\frac{2{\rm Im}w_0}{|w_0|^2+1},\frac{|w_0|^2-1}{|w_0|^2+1}\right)} & {[1:w_0]}
                    \arrow[from=1-2, to=1-3]
                    \arrow[maps to, from=2-2, to=2-3]
                    \arrow[maps to, from=3-3, to=3-2]
                \end{tikzcd}
            \end{center}
            以及
            \begin{center}
                \begin{tikzcd}[column sep=large,row sep=tiny]
                    {} & {S^2\backslash\{S\}} & {U_1} & {} \\
                    {} & {(x,y,z)} & {\left[\frac{x+\sqrt{-1}y}{1+z}:1\right]} & {} \\
                    & {\left(\frac{2{\rm Re}w_1}{|w_1|^2+1},\frac{2{\rm Im}w_1}{|w_1|^2+1},\frac{1-|w_1|^2}{1+|w_1|^2}\right)} & {[w_1:1]}
                    \arrow[from=1-2, to=1-3]
                    \arrow[maps to, from=2-2, to=2-3]
                    \arrow[maps to, from=3-3, to=3-2]
                \end{tikzcd}
            \end{center}
            注意到当 $(x,y,z)\in S^2$ 时, $\left[1:\frac{x+\sqrt{-1}y}{1-z}\right] = \left[\frac{x+\sqrt{-1}y}{1+z}:1\right]$, 
            因此同构映射 $S^2\backslash\{N,S\}\rightarrow U_0\cap U_1$ 是良定义的. 从而有 $\mCP^1\cong S^2$.
        \end{example}
            
        \begin{example}[齐次多项式零点定义的射影空间的子集]
            设 $f(\ft{z}{0}{n})$ 是 $\mC^{n+1}$ 上的 $d$ 次齐次多项式, 将其限制在 $\mC^{n+1}\backslash\{0\}$ 上.
            设 $0$ 是限制后函数的正则值, 则 $Z(f):=f^{-1}(0)$ 是复流形, 且若 $\omega = (\ft{\omega}{0}{n})\in Z(f)$, 
            则对 $\forall\,\lambda\in\mC^{*}$, $\lambda\omega = (\ft{\lambda\omega}{0}{n})\in Z(f)$.
             于是有 $\mC^*$ 在 $Z(f)$ 上的作用, 定义:
            \begin{equation*}
                V(f):=Z(f)\big/ \mC^*\subset\left(\mC^{n+1}\backslash\{0\}\right)\big/\mC^* = \mathbb{C}\mathrm{P}^n.
            \end{equation*}
            
            下面说明 $V(f)$ 仍有复流形结构, 我们知道复射影空间有典范的坐标覆盖: 
            \begin{equation*}
                \mathbb{C}\mathrm{P}^n = \bigcup_{i=0}^{n}U_i = \left\{[z_0:\cdots:z_n]\,\big|\,z_i\neq0\right\}.
            \end{equation*}
            坐标映射为
            \begin{align*}
                \varphi_i:U_i&\rightarrow\mC^n, \\
                [z_0:\cdots:z_n] = \left[\cdots:\frac{z_{i-1}}{z_i}:1:\frac{z_{i+1}}{z_i}:\cdots\right]&\mapsto\left(\dots,\frac{z_{i-1}}{z_i},\frac{z_{i+1}}{z_i},\dots\right).
            \end{align*}
            则
            \begin{equation*}
                V(f) = \bigcup_{i=0}^{n}\left(U_i\cap V(f)\right),
            \end{equation*}
            且 
            \begin{align*}
                U_i\cap V(f) &= \left\{[z_0:\cdots:z_n]\,\big|\,f(\ft{z}{0}{n}) = 0,\,z_i\neq0\right\} \\
                & = \left\{[\cdots:z_{i-1}:1:z_{i+1}:\cdots]\,\big|\,f(\dots,z_{i-1},1,z_{i+1},\dots) = 0\right\}.
            \end{align*}
            于是
            \begin{align*}
                \varphi_i\left(U_i\cap V(f)\right) = \left\{(\ft{z}{1}{n})\in\mC^n\,\big|\,f(z_1,\dots,z_i,1,z_{i+1},\dots,z_n)=0\right\}. \\
            \end{align*}
            令 $f_i(\ft{z}{1}{n}) = f(z_1,\dots,z_i,1,z_{i+1},\dots,z_n)$, 则 $\varphi_i\left(U_i\cap V(f)\right)$ 是 $f_i$ 的零点集, 且
            \begin{equation*}
                \pd{f_i}{z_j}(\ft{z}{1}{n}) =
                \begin{cases}
                    \pd{f}{z_{j-1}}(z_1,\dots,z_{i},1,z_{i+1},\dots,z_n),&\quad j\leqslant i, \\
                    \pd{f}{z_{j}}(z_1,\dots,z_{i},1,z_{i+1},\dots,z_n),&\quad j > i.
                \end{cases}
            \end{equation*}
            假设 $0$ 不是 $f_i$ 的正则值, 即存在 $\omega = (\ft{\omega}{1}{n})$ 使得
            \begin{equation*}
                \left\{
                    \begin{aligned}
                        &f_i(\omega) = 0, \\
                        &\pd{f_i}{z_j}(\omega) = 0,\;\forall\,1\leqslant j\leqslant n.
                    \end{aligned}
                \right.
            \end{equation*}
            因为 $f$ 是 $d$ 次齐次多项式, 于是 
            \begin{equation*}
                f(\ft{\lambda z}{0}{n}) = \lambda^{d}f(\ft{z}{0}{n})
            \end{equation*}
            两侧对 $\lambda$ 求导并令 $\lambda = 1$ 得 
            \begin{equation*}
                z_0\pd{f}{z_0}(z)+\cdots+z_n\pd{f}{z_n}(z) = d\cdot f(z)
            \end{equation*}
            代入 $z = (\omega_1,\dots,\omega_{i},1,\omega_{i+1},\dots,\omega_n)$, 由前文可知
            \begin{equation*}
                \left\{
                    \begin{aligned}
                        &f(\omega_1,\dots,\omega_{i},1,\omega_{i+1},\dots,\omega_n) = 0, \\
                        &\pd{f}{z_j}(\omega_1,\dots,\omega_{i},1,\omega_{i+1},\dots,\omega_n) = 0,\;j\neq i.
                    \end{aligned}
                \right.
            \end{equation*}
            且
            \begin{align*}
                &\pd{f}{z_i}(\omega_1,\dots,\omega_{i},1,\omega_{i+1},\dots,\omega_n) \\
                =& d\cdot f(\omega_1,\dots,\omega_{i},1,\omega_{i+1},\dots,\omega_n) - \sum_{j \neq i}\pd{f}{z_j}(\omega_1,\dots,\omega_{i},1,\omega_{i+1},\dots,\omega_n) \\
                =& 0.
            \end{align*}
            则 $(\omega_1,\dots,\omega_{i},1,\omega_{i+1},\dots,\omega_n)$ 不是 $f$ 的正则点, $0$ 不是 $f$ 的正则值, 矛盾!
            
            因此每个 $\varphi_i\left(U_i\cap V(f)\right)$ 都是复流形, 因此 $V(f)$ 是一个复流形, 且注意到 $V(f)$ 是 $\mCP^n$ 的闭子集, 从而也是一个紧集.
        \end{example}

    \section{K\"ahler几何}
    \begin{definition}
        Let $M$ be an almost complex manifold, $J$ be its almost complex structure. We say a metric $g$ is a Hermitian metric if it satisfies
        \[
            g(X,Y)=g(JX,JY)
        \]
    \end{definition}
    We then extend it to be a $\mC$-bilinear tensor, i.e. $g_{\mC}\in C^{\infty}(M,\bigwedge^{\!2}T^*_{\mC}M)$. 

    Now let $M$ be a complex manifold, $J$ be its natural complex structure. In a local coordinate $\{U;(z^1,\cdots z^n)\}$, let
    \[
        g_{i\bar{j}}:=\left\langle \pd{}{z^i},\pd{}{\bar{z}^j}\right\rangle 
    \]
    It satisfies
    \begin{gather*}
        g_{i\bar{j}}=g_{\bar{j}i} \\
        g_{j\bar{i}}=\overline{g_{i\bar{j}}}
    \end{gather*}
    The first equality comes from the symmetirc of $g$. The second one comes from:
    \begin{align*}
    &g_{j\bar{i}} = \left\langle\pd{}{z^j},\pd{}{\bar{z}^i}\right\rangle = \left\langle\frac{1}{2}\left(\pd{}{x^j}-\sqrt{-1}\pd{}{y^j}\right), \frac{1}{2}\left(\pd{}{x^i}+\sqrt{-1}\pd{}{y^i}\right)\right\rangle \\
    =& \frac{1}{4} \left[\left(\left\langle\pd{}{x^j},\pd{}{x^i}\right\rangle+\left\langle\pd{}{y^j},\pd{}{y^i}\right\rangle\right)+\sqrt{-1}\left(\left\langle\pd{}{x^j},\pd{}{y^i}\right\rangle-\left\langle\pd{}{y^j},\pd{}{x^i}\right\rangle\right)\right] \\
    =& \frac{1}{4} \left[\left(\left\langle\pd{}{x^i},\pd{}{x^j}\right\rangle+\left\langle\pd{}{y^i},\pd{}{y^j}\right\rangle\right)-\sqrt{-1}\left(\left\langle\pd{}{x^i},\pd{}{y^j}\right\rangle-\left\langle\pd{}{x^j},\pd{}{y^i}\right\rangle\right)\right] \\
    =& \overline{\left\langle\frac{1}{2}\left(\pd{}{x^i}-\sqrt{-1}\pd{}{y^i}\right), \frac{1}{2}\left(\pd{}{x^j}+\sqrt{-1}\pd{}{y^j}\right)\right\rangle} = \overline{\left\langle\pd{}{z^i},\pd{}{\bar{z}^j}\right\rangle} = \overline{g_{i\bar{j}}}
    \end{align*}