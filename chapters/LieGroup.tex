\chapter{Lie群基础}
\section{Lie群同态}
    \begin{theorem}
        设 $G,\,H$ 为 $\mathrm{Lie}$ 群, 且 $G$ 连通, 它们的 $\mathrm{Lie}$ 代数分别为 $\mathfrak{g,\,h}$. 现有两个 $\mathrm{Lie}$ 群同态 $\varphi,\,\psi:G\rightarrow H$, 
        若 $\varphi,\,\psi$ 诱导的 $\mathrm{Lie}$ 代数同态 $\varphi_*,\,\psi_*:\mathfrak{g}\rightarrow\mathfrak{h}$ 是相同的, 即 $\varphi_*=\psi_*$, 那么 $\varphi=\psi$.
    \end{theorem}
    \begin{proof}
        设 $\omega_1,\dots,\omega_n$ 是 $\mathfrak{h}$ 的一组基, 也即 $H$ 上处处线性无关的一组左不变一次微分式.
        并分别记 $\pi_1:G\times H\rightarrow G$, $\pi_2:G\times H\rightarrow H$ 为自然投影, 则可验证
        \begin{equation*}
            \left\{\pi_1^*\varphi^*\omega_i-\pi_2^*\omega_i\,\Big|\,i=1,\dots,n\right\} = \left\{\pi_1^*\psi^*\omega_i-\pi_2^*\omega_i\,\Big|\,i=1,\dots,n\right\}
        \end{equation*}
        张成的理想都是 $G\times H$ 上的左不变(暂时不知道有什么用)微分理想, 且 $\varphi(e) = \psi(e) = e$, 由微分理想的结果再加上 $G$ 是连通的可知 $\varphi = \psi$.
    \end{proof}
\section{Lie子群}
    \begin{definition}
        若 $(H,\,\varphi)$ 满足:
        \begin{itemize}
            \item $H$ 是一个 $\mathrm{Lie}$ 群.
            \item $\varphi:H\rightarrow G$ 是微分流形的浸入
            \item $\varphi:H\rightarrow G$ 是群同态
        \end{itemize}
        则称 $(H,\,\varphi)$ 为 $\mathrm{Lie}$ 群 $G$ 的 $\mathrm{Lie}$ 子群.
    \end{definition}
    \begin{remark}
        我们可以定义 $\mathrm{Lie}$ 子群之间的等价(就像浸入子流形的等价一样),
        并且可以在每个等价类中选取 $(H,\,i)$ 使得 $H\subset G$ 是 $G$ 的子集(但 $H$ 的拓扑不一定是 $G$ 的相对拓扑), 含入映射 $i:H\hookrightarrow G$ 是微分流形的浸入.
        此时 $\mathfrak{h}$ 也可自然看成 $\mathfrak{g}$ 的子集.
    \end{remark}
    \begin{theorem}[Lie子代数与连通Lie子群的一一对应]
        设 $G$ 为 $\mathrm{Lie}$ 群, 它的 $\mathrm{Lie}$ 代数为 $\mathfrak{g}$. 设 $\mathfrak{h}\subset\mathfrak{g}$ 为 $\mathrm{Lie}$ 子代数,
        则存在唯一的连通 $\mathrm{Lie}$ 子群 $H$ 使得 $H$ 的 $\mathrm{Lie}$ 代数就是 $\mathfrak{h}$.
    \end{theorem}
    \begin{proof}
        $\mathfrak{h}$ 对应 $G$ 上一个对合分布 $\mathcal{D}$, 记 $(H,\,\varphi)$ 为 $\mathcal{D}$ 的经过单位元 $e$ 的极大积分子流形, 则 $(H,\,\varphi)$ 即为所求.
        
        任取 $\sigma\in H$, 则 $(H,\,l_{\sigma^{-1}}\circ\varphi)$ 仍为 $\mathcal{D}$ 的积分子流形(因为 $\mathcal{D}$ 左平移不变), 再由 $(H,\varphi)$ 的极大性, 可以推出 $H$ 是抽象子群.
        
        证明 $H$ 的群结构与微分结构相容(即 $(\sigma,\tau)\mapsto\sigma\tau$ 是 $H\times H$ 到 $H$ 的光滑映射) 和 $H$ 的唯一性则需要极大积分子流形的相关结果.
    \end{proof}
    \begin{theorem}[闭Lie子群与正则子流形的关系]
        设 $(H,\,\varphi)$ 是 $G$ 的 $\mathrm{Lie}$ 子群, 则以下两条
        \begin{itemize}
            \item $\varphi$ 是嵌入, 即 $\varphi$ 是 $H$ 到 $\varphi(H)$(取关于 $G$ 的相对拓扑)的同胚.
            \item $(H,\,\varphi)$ 是 $G$ 的闭 $\mathrm{Lie}$ 子群($\varphi(H)$ 是 $G$ 的闭子集).
        \end{itemize}
        是等价的.
    \end{theorem}
    \begin{proof}
        证明暂时没看懂
    \end{proof}

\section{覆叠映射}

    若 $\pi:\tilde{M}\rightarrow M$ 是连通局部道路联通空间 $\tilde{M}$ 到 $M$ 的覆叠映射, 则 $\tilde{M}$ $\mathrm{Hausdorff}$、第二可数且局部同胚于欧式空间。
    而且 $\tilde{M}$ 上存在唯一的微分结构使得 $\pi$ 是光滑且局部同胚映射。

    若 $G$ 为 $\mathrm{Lie}$ 群, 则存在单连通空间 $\tilde{G}$ 以及覆叠映射 $\pi:\tilde{G}\rightarrow G$, 由上述结果可知能在 $\tilde{G}$ 上定义合适的光滑结构,
    更进一步地, 能在 $\tilde{G}$ 上定义群结构使得其成为 $\mathrm{Lie}$ 群, 且 $\pi$ 成为 $\mathrm{Lie}$ 群同态.

    \begin{proposition}[Lie群同态与覆叠映射]
        设 $G$ 和 $H$ 为连通 $\mathrm{Lie}$ 群, 且有 $\mathrm{Lie}$ 群同态 $\varphi:G\rightarrow H$.
        则 $\varphi$ 为覆叠映射当且仅当切映射 $\varphi_*:G_e\rightarrow H_e$ 是线性同构.
    \end{proposition}

\section{单连通Lie群}
    \begin{theorem}
        敬请期待.
    \end{theorem}
