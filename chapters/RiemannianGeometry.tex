\chapter{黎曼几何}
\section{仿射联络}
    在局部坐标 $\left(U;x^i\right)$ 下有
    \begin{equation*}
        D_{\pd{}{x^i}}\pd{}{x^j}=\Gamma^{k}_{ji}\pd{}{x^k}
    \end{equation*}
    其中 $\Gamma^{k}_{ji}$ 称为 $D$ 在局部坐标下的\textbf{联络系数}.

    我们进一步定义任意 $(r,s)$ 型张量的协变导数, 首先定义1阶微分形式 $\alpha$ 的协变导数:
    \begin{align*}
        \left(D_{X}\alpha\right)(Y) &= C^1_1\left((D_X\alpha)\otimes Y\right) \\
        &= C^1_1\left(D_X(\alpha\otimes Y)-\alpha\otimes(D_XY)\right) \\
        &= X(\alpha(Y))-\alpha(D_XY)
    \end{align*}
    特别地,
    \begin{equation*}
        D_{\pd{}{x^i}}{\md x^j} = -\Gamma_{ki}^j{\md x^k}
    \end{equation*}

    对于一般的 $(r,s)$ 型张量 $\tau\in\mathcal{T}^r_s(M)$, 定义它沿向量场 $X$ 的协变导数为
    \begin{align*}
        &(D_X\tau)(\alpha^1,\dots,\alpha^r,X_1,\dots,X_s) \\
        =& X\left(\tau(\alpha^1,\dots,\alpha^r,X_1,\dots,X_s)\right) \\
        &-\sum_{a=1}^{r}\tau\left(\alpha^1,\dots,D_X\alpha^a,\dots,\alpha^r,X_1,\dots,X_s\right) \\
        &-\sum_{b=1}^{s}\tau\left(\alpha^1,\dots,\alpha^r,X_1,\dots,D_X{X^b},\dots,X_s\right)
    \end{align*}

    定义一个 $(r,s)$ 型张量场 $\tau$ 沿向量场 $X$ 的协变微分为
    \begin{equation*}
        (D\tau)(\alpha^1,\dots,\alpha^r,X_1,\dots,X_s,X):=(D_X\tau)(\alpha^1,\dots,\alpha^r,X_1,\dots,X_s)
    \end{equation*}
    可以看到 $D$ 把 $\tau$ 变为一个 $(r,s+1)$ 型张量. 在局部坐标 $\left(U;x^i\right)$ 下的分量表达式为
    \begin{align*}
        \tau^{i_1\cdots i_r,i}_{j_1\cdots j_s} =& \pd{\tau^{i_1\cdots i_r}_{j_1\cdots j_s}}{x^i} + \sum_{a=1}^{r}\tau^{i_1\cdots i_{a-1}k i_{a+1}\cdots i_r}_{j_1\cdots j_s}\Gamma^{i_a}_{ki} \\
        & -\sum_{b=1}^{s}\tau^{i_1\cdots i_r}_{j_i\cdots j_{b-1}k j_{b+1}\cdots j_s}\Gamma^{k}_{j_b i}
    \end{align*}
\section{黎曼联络}
    定义挠率张量 $T$ 为
    \begin{equation*}
        T(X,Y)=D_XY-D_XY-[X,Y]
    \end{equation*}
    在局部坐标 $\left(U;x^i\right)$ 下 $T$ 的表达式为
    \begin{equation*}
        T = \left(\Gamma^{k}_{ji}-\Gamma^{k}_{ij}\right)\pd{}{x^k}\otimes\md x^i\otimes\md x^j
    \end{equation*}
    若由联络定义的挠率张量 $T$ 恒等于零, 则称该联络是\textbf{无挠联络}. 由局部坐标表达式可知无挠联络的联络系数满足 $\Gamma^{k}_{ji}=\Gamma^{k}_{ij}$.

    若联络 $D$ 和 度量 $g$ 满足 $g$ 的协变微分 $Dg\equiv0$, 则称联络和度量是\textbf{相容的}. 该条件等价于 $(D_Zg)(X,Y)=0,\;\forall X,Y,Z\in\mathfrak{X}(M)$ 因为
    \begin{equation*}
        (D_Zg)(X,Y) = Z\left\langle X,Y\right\rangle - \left\langle D_ZX,Y\right\rangle - \left\langle X,D_ZY\right\rangle
    \end{equation*}
    所以联络和度量相容当且仅当
    \begin{equation*}
        Z\left\langle X,Y\right\rangle = \left\langle D_ZX,Y\right\rangle + \left\langle X,D_ZY\right\rangle
    \end{equation*}

\subsection{黎曼几何基本定理}
    \begin{theorem}
        设 $(M,g)$ 是黎曼流形, 则 $M$ 上存在唯一一个与 $g$ 相容的无挠联络 $D$, 我们称之为\textbf{黎曼联络}.
    \end{theorem}

    \begin{theorem}[Koszul公式] \label{eq:koszul}
        若联络 $D$ 满足无挠且与度量 $\left\langle \cdot,\cdot\right\rangle$ 相容, 则有公式
        \begin{align}
            2\left\langle D_XY,Z\right\rangle =& X\left\langle Y,Z\right\rangle+ Y\left\langle X,Z\right\rangle -Z\left\langle X,Y\right\rangle \nonumber \\
            & +\left\langle [X,Y],Z\right\rangle -\left\langle X,[Y,Z]\right\rangle -\left\langle Y,[X,Z]\right\rangle
        \end{align}
    \end{theorem}
    利用Koszul公式可以很容易证明黎曼几何基本定理.

\section{黎曼联络系数的坐标变换}
    \begin{equation*}
        \Gamma_{ij}^{k}=\tilde{\Gamma}_{pq}^{r}\frac{\partial\tilde{x}^p}{\partial x^i}\frac{\partial\tilde{x}^q}{\partial x^j}\frac{\partial x^k}{\partial\tilde{x}^r}+\frac{\partial^2\tilde{x}^r}{\partial x^i \partial x^j}\frac{\partial x^k}{\partial\tilde{x}^r}
    \end{equation*}
    因为
    \begin{align*}
        0 =& \frac{\partial}{\partial x^i}\left(\frac{\partial x^k}{\partial\tilde{x}^r}\frac{\partial\tilde{x}^r}{\partial x^j}\right) \\
        =& \frac{\partial}{\partial x^i}\left(\frac{\partial x^k}{\partial\tilde{x}^r}\right)\frac{\partial\tilde{x}^r}{\partial x^j} + \frac{\partial x^k}{\partial\tilde{x}^r}\frac{\partial^2\tilde{x}^r}{\partial x^i\partial x^j} \\
        =& \frac{\partial\tilde{x}^s}{\partial x^i}\frac{\partial}{\partial\tilde{x}^s}\left(\frac{\partial x^k}{\partial\tilde{x}^r}\right)\frac{\partial\tilde{x}^r}{\partial x^j} + \frac{\partial x^k}{\partial\tilde{x}^r}\frac{\partial^2\tilde{x}^r}{\partial x^i\partial x^j} \\
        =& \frac{\partial\tilde{x}^s}{\partial x^i}\frac{\partial^2 x^k}{\partial\tilde{x}^r\partial\tilde{x}^s}\frac{\partial\tilde{x}^r}{\partial x^j} + \frac{\partial x^k}{\partial\tilde{x}^r}\frac{\partial^2\tilde{x}^r}{\partial x^i\partial x^j} \\
    \end{align*}
    故
    \begin{equation*}
        \frac{\partial^2 x^k}{\partial\tilde{x}^r\partial\tilde{x}^s}\frac{\partial\tilde{x}^s}{\partial x^i}\frac{\partial\tilde{x}^r}{\partial x^j} = -\frac{\partial^2\tilde{x}^r}{\partial x^i\partial x^j}\frac{\partial x^k}{\partial\tilde{x}^r}\Big(\text{一般而言}\neq0\Big)
    \end{equation*}

\section{由联络定义的各种微分算子}
    \begin{itemize}
        \item \textbf{散度算子}div: $\mathrm{div}(X)=C_1^1(DX)$
        \begin{equation*}
            (\mathrm{div}X)|_U=X^i_{,i}=\pd{X^i}{x^i}+X^k\Gamma^i_{ki}=\frac{1}{\sqrt{G}}\pd{}{x^i}(\sqrt{G}X^i).
        \end{equation*}
        \item \textbf{梯度算子}$\nabla$: $\left\langle \nabla f,X\right\rangle:=\md{f}(X)=X(f)$ 
        \begin{equation*}
            (\nabla f)|_U = f^i\pd{}{x^i}=f_jg^{ij}\pd{}{x^i}=g^{ij}\pd{f}{x^j}\pd{}{x^i}
        \end{equation*}
        \item \textbf{Laplace算子} $\Delta$: $\Delta:=\mathrm{div}\circ\nabla$
        \begin{equation*}
            (\Delta f)|_U = \frac{1}{\sqrt{G}}\pd{}{x^i}\left(\sqrt{G}g^{ij}\pd{f}{x^j}\right)
        \end{equation*}
        \item \textbf{函数的Hessian算子} $\mathrm{Hess}(f):=D(Df)=D(\md f)$
        \begin{equation*}
            (\mathrm{Hess}(f))(X,Y)=Y(X(f))-(D_YX)(f)
        \end{equation*}
        分量的局部坐标表达式为
        \begin{equation*}
            \mathrm{Hess}(f)_{ij}=\pd{}{x^j}\left(\pd{f}{x^i}\right)-\Gamma^{k}_{ij}\pd{f}{x^k}=f_{i,j}
        \end{equation*}
        Hessian算子与Laplace算子的关系是
        \begin{equation*}
            \Delta f=\mathrm{tr}_g(\mathrm{Hess}(f))=\mathrm{tr}(\mathrm{Hess}(f))
        \end{equation*}
        用分量表示即
        \begin{equation*}
            \Delta f = g^{ij}f_{i,j}
        \end{equation*}
        \item \textbf{Hodge星 $*$ 算子}: 设 $\omega\in A^r(M)$ 且
        \begin{equation*}
            \omega|_U=\frac{1}{r!}\omega_{i_1}\md{x^{i_1}}\wedge\cdots\wedge\md{x^{i_r}}
        \end{equation*}
        命
        \begin{equation*}
            (*\omega)|_U=\frac{\sqrt{G}}{r!(m-r)!}\delta^{1\cdots m}_{i_1\cdots i_m}\omega^{i_1\cdots i_r}\md{x^{i_{r+1}}}\wedge\cdots\wedge\md{x^{i_m}}
        \end{equation*}
        \item \textbf{余微分算子}$\delta := (-1)^{mr+1}*\circ\md\circ*=\md^*$
        \begin{equation*}
            (\md{\varphi},\psi)=(\varphi, \delta\psi)
        \end{equation*}
        \item \textbf{Hodge-Laplace算子}$\bar{\Delta}:=\md\circ\delta+\delta\circ\md$
        \begin{equation*}
            \bar{\Delta}f=\Delta f
        \end{equation*}
    \end{itemize}
\section{不同双曲模型之间的等距同构}
\subsection{三种双曲模型}
    \begin{itemize}
        \item Poincare上半平面模型 $\mathbb{H}^n:=\left\{\left(x^1,\dots,x^n\right)\in\mathbb{R}^n\,\Big|\,x^n>0\right\}$  
        
        其上的度量定义为
        \begin{equation*}
            h=\frac{(\md x^1)^2+\cdots+(\md x^n)^2}{(x^n)^2}
        \end{equation*}

        \item Poincare球模型 $D^n:=\left\{\left(y^1,\dots,y^n\right)\in\mathbb{R}^n\,\Big|\,|y|<1\right\}$
        
        其上的度量定义为
        \begin{equation*}
            g_{-1}=\frac{(\md y^1)^2+\cdots+(\md y^n)^2}{(1-|y|^2)^2}
        \end{equation*}

        \item Minkowski模型: $\mathbb{R}^{n+1}$ 上定义Lorentz内积 $l=\langle\cdot,\cdot\rangle$
        \begin{equation*}
            l(x,y)=\langle x,y\rangle_1=\sum_{i=1}^{n}x^iy^i-x^{n+1}y^{n+1},\quad\forall x,\,y\in\mathbb{R}^n,
        \end{equation*}
        考虑 $\mathbb{R}^{n+1}$ 中双叶双曲面的上半叶
        \begin{equation*}
            M^n:=\left\{z=\left(z^1,\dots,z^{n+1}\right)\in\mathbb{R}^{n+1}\,\Big|\,\langle z,z\rangle=-1\,\text{且}\,z^{n+1}>0\right\},
        \end{equation*}
        其上的度量定义为嵌入映射 $i$ 诱导的拉回度量 $m=i^*l$.
    \end{itemize}
\subsection{双曲模型间的等距同构}
    \begin{itemize}
        \item $M^n$ 与 $D^n$ 之间: 由球极投影给出
        \begin{align*}
            \varphi:M^n&\rightarrow D^n \\
            \left(z^1,\dots,z^{n+1}\right)&\mapsto\left(y^1,\dots,y^n\right)=\left(\frac{z^1}{1+z^{n+1}},\dots,\frac{z^n}{1+z^{n+1}}\right) \\
            \varphi^{-1}:D^n&\rightarrow M^n \\
            \left(y^1,\dots,y^n\right)&\mapsto\left(z^1,\dots,z^{n+1}\right)=\left(\frac{2y^1}{1-|y|^2}\cdots\frac{2y^n}{1-|y|^2},\frac{1+|y|^2}{1-|y|^2}\right)
        \end{align*}
        \item $\mathbb{H}^n$ 与 $D^n$ 之间: 由分式线性变换的推广, Cayley变换给出
        \begin{align*}
            \psi:\mathbb{H}^n&\rightarrow D^n \\
            \mathbb{R}^{n-1}\times\mathbb{R}\ni\left(\vec{x},x^n\right)&\mapsto\left(\vec{y},y^n\right)=\left(\frac{2\vec{x}}{|\vec{x}|^2+(x^n+1)^2},\frac{|\vec{x}|^2+(x^n)^2-1}{|\vec{x}|^2+(x^n+1)^2}\right) \\
            \psi^{-1}:D^n&\rightarrow\mathbb{H}^n \\
            \mathbb{R}^{n-1}\times\mathbb{R}\ni\left(\vec{y},y^n\right)&\mapsto\left(\vec{x},x^n\right)=\left(\frac{2\vec{y}}{|\vec{y}|^2+(y^n-1)^2},\frac{1-|\vec{y}|^2-(y^n)^2}{|\vec{y}|^2+(y^n-1)^2}\right)
        \end{align*}
    \end{itemize}
    可以试着用它们之间的同构给出 $\mathrm{SO}(2,1)$ 到 $\mathrm{SL}(2,\mathbb{R})$ 的显式同构.

\section{黎曼几何中的各种曲率}
\subsection{曲率张量}
\subsubsection{缘起}
    在协变微分中我们引进记号
    \begin{equation*}
        \nabla Z(X) :=\nabla_X{Z}
    \end{equation*}
    于是可以考虑多次协变微分是否可换序的问题, 即
    \begin{equation*}
        \nabla^2Z(X,Y) \stackrel{?}{=} \nabla^2Z(Y,X)
    \end{equation*}
    我们仔细地将等号两侧的式子展开
    \begin{align*}
        \nabla^2Z(X,Y) &= (\nabla_Y\nabla Z)(X) \\
        &= \nabla_Y((\nabla Z)(X)) - (\nabla Z)(\nabla_Y{X}) \\
        &= \nabla_Y\nabla_X{Z}-\nabla_{\nabla_Y{X}}Z
    \end{align*}
    若该仿射联络是无挠联络, 则有
    \begin{align*}
        & \nabla^2Z(X,Y) - \nabla^2Z(Y,X) \\
        =& \nabla_Y\nabla_X{Z}-\nabla_X\nabla_Y{Z}-\nabla_{\nabla_Y{X}-\nabla_X{Y}}Z \\
        =& \nabla_Y\nabla_X{Z}-\nabla_X\nabla_Y{Z}-\nabla_{[Y,X]}Z
    \end{align*}
    受此启发, 我们定义曲率张量 $\mathcal{R}(\cdot,\cdot)\cdot:\mathfrak{X}(M)\times\mathfrak{X}(M)\times\mathfrak{X}(M)\rightarrow\mathfrak{X}(M)$ 为:
    \begin{definition}
        $\mathcal{R}(X,Y)Z := \nabla_X\nabla_Y{Z} - \nabla_Y\nabla_X{Z} - \nabla_{[X,Y]}Z$
    \end{definition}
    
    \begin{remark}
        仿射联络空间 $(M,D)$ 的挠率张量和曲率张量实际上是判断它偏离仿射空间的量度.
    \end{remark}

\subsubsection{Ricci恒等式}
回顾 \textbf{缘起} 中的内容, $Z$ 作为一个(1,0)型张量场可自然与一个(1,0)型张量场(即一次微分式) $\omega$ 做配合. 因为两次协变微分后 $\nabla2{Z}$ 为一个(1,2)型张量. 
那么一方面我们有
\begin{equation*}
    \nabla^2Z(\omega,Y,X) - \nabla^2Z(\omega,X,Y)
\end{equation*}
另一方面我们有
\begin{equation*}
    (\mathcal{R}(X,Y)Z)(\omega) = (\nabla_X\nabla_Y{Z} - \nabla_Y\nabla_X{Z}-\nabla_{[X,Y]}{Z})(\omega)
\end{equation*}
经过一番激烈地运算会发现

\begin{gather*}
    \nabla^2Z(\omega,Y,X) - \nabla^2Z(\omega,X,Y) = Z(\mathcal{R}(Y,X)\omega) \\
    (\mathcal{R}(X,Y)Z)(\omega) = -Z(\mathcal{R}(X,Y)\omega) = Z(\mathcal{R}(Y,X)\omega)
\end{gather*}
在这个角度下它们确实是一样的.
\begin{definition}[曲率算子作用在张量场]
    我们先推导出曲率算子 $\mathcal{R}(X,Y)$ 作用在 $(0,0)$ 型(即光滑函数)、$(1,0)$ 型(即光滑向量场)、$(0,1)$ 型(即一次微分式)张量上是怎么样的. 然后设曲率算子满足:
    \begin{itemize}
        \item 与张量积运算满足Leibniz法则
        \item 与缩并运算可交换
    \end{itemize}
    以此定义曲率算子在一般 $(r,s)$ 型张量上 $\tau$ 的作用.
\end{definition}
\begin{proposition}[Ricci恒等式]
    设 $\tau$ 为 $(r,s)$ 型张量场, $\omega^1,\dots,\omega^r$ 是 $r$ 个光滑1形式, $X_1,\dots,X_s$ 是 $s$ 个光滑向量场, 设 $Y,\,Z$ 为两个光滑向量场, 则
    \begin{align*}
        & \nabla^2\tau(\cdots,\omega^i,\cdots;\cdots,X_j,\cdots;Z,Y) - \nabla^2\tau(\cdots,\omega^i,\cdots;\cdots,X_j,\cdots;Y,Z) \\
        =& \tau(\cdots,\mathcal{R}(Z,Y)\omega^i,\cdots;\cdots,X_j,\cdots) + \tau(\cdots,\omega^i,\cdots;\cdots,\mathcal{R}(Z,Y)X_j,\cdots) \\
        =&(\mathcal{R}(Y,Z)\tau)(\cdots,\omega^i,\cdots;\cdots,X_j,\cdots) \;(\text{注意}Y,Z\text{顺序})
    \end{align*}
\end{proposition}

\subsection{黎曼曲率张量的代数性质}
\subsubsection{对称性}
    设 $V$ 为 $n$ 维线性空间, 若 $R\in(V^*)^{\otimes4}$ 且满足:
    \begin{enumerate}
        \item $R(X,Y,Z,W) = -R(Y,X,Z,W) = -R(X,Y,W,Z) = R(Z,W,X,Y)$
        \item $R(X,Y,Z,W) + R(Y,Z,X,W) + R(Z,X,Y,W) = 0$
    \end{enumerate} 对 $\forall X,\,Y,\,Z,\,W\in V$ 均成立.
    则称 $R$ 为代数黎曼曲率张量, 全体代数黎曼曲率张量构成的线性空间记为 $\mathcal{R}(V^*)$.

    我们分别记 $\Sigma^n(V^*),\,\bigwedge^n(V^*)$ 为对偶空间 $V^*$ 的对称张量积和反对称张量积, 则由第一条可知 $\mathcal{R}(V^*)\subset\Sigma^2(\bigwedge^2V^*)$.
    容易看出 $\bigwedge^4V^*$ 也为 $\Sigma^2(\bigwedge^2V^*)$ 的一个子空间, 而且若在 $V$ 上有度量 $\langle \cdot,\cdot\rangle_g$, 由第二条可推出
    \begin{equation*}
        \textstyle \Sigma^2(\bigwedge^2V^*) = \mathcal{R}(V^*)\oplus\bigwedge^4V^*
    \end{equation*}
    直和项互相正交.

    我们分两步说明上述论断. 首先对 $\forall R\in\mathcal{R}(V^*),\,S\in\bigwedge^4V^*$, 设 $e_i$ 为 $V$ 的一组标准正交基, 则
    \begin{align*}
        & \langle R,S\rangle_g \\
        = & \sum_{i,j,k,l}R(e_i,e_j,e_k,e_l)S(e_i,e_j,e_k,e_l) \\
        = & \frac{1}{3}\sum_{i,j,k,l}\left(R(e_i,e_j,e_k,e_l) + R(e_j,e_k,e_i,e_l) + R(e_k,e_i,e_j,e_l)\right)S(e_i,e_j,e_k,e_l) \\
        = & 0
    \end{align*}

    其次, 对每一个 $T\in\Sigma^2(\bigwedge^2V^*)\subset(V^*)^{\otimes4}$, 我们原本就有一个将一般(0,4)型张量化为反对称张量的算子 $\mathcal{A}$, 在附加的对称性下, 反对称算子在 $\Sigma^2(\bigwedge^2V^*)$ 上的作用可以写为
    \begin{equation*}
        \mathcal{A}(T)(X,Y,Z,W) = \frac{1}{3}\left(T(X,Y,Z,W) + T(Y,Z,X,W) + T(Z,X,Y,W)\right)
    \end{equation*}
    可以验证这样定义的 $\mathcal{A}(T)$ 确实为一个反对称张量:
    \begin{align*}
        & \mathcal{A}(T)(Y,Z,W,X) \\
        =& \frac{1}{3}\left(T(Y,Z,W,X) + T(Z,W,Y,X) + T(W,Y,Z,X)\right) \\
        =& -\frac{1}{3}\left(T(X,Y,Z,W) + T(Y,Z,X,W) + T(Z,X,Y,W)\right) \\
        =& -\mathcal{A}(T)(X,Y,Z,W)
    \end{align*}
    \begin{remark}
        因为 $S_4 = \langle(12),(1234)\rangle$ 而 $(12),\,(1234)$ 在 $\mathcal{A}(T)$ 上的作用均符合反对称张量符号变化规律, 所以 $\mathcal{A}(T)$ 是反对称张量.
    \end{remark}

    得到反对称张量的分量后就可以很容易得到代数黎曼曲率张量的分量了:
    \begin{equation*}
        T_R := T - \mathcal{A}(T)
    \end{equation*}
    可以验证 $T_R\in\mathcal{R}(V^*)$
    \begin{align*}
        & T_R(X,Y,Z,W) + T_R(Y,Z,X,W) + T_R(Z,X,Y,W) \\
        =& T(X,Y,Z,W) + T(Y,Z,X,W) + T(Z,X,Y,W) - 3\mathcal{A}(T) \\
        =& 0
    \end{align*}

    \begin{proposition}
        $n$ 维线性空间 $V$ 上的代数黎曼曲率张量的维数为 $\displaystyle \frac{n^2(n^2-1)}{12}$
    \end{proposition}
    \begin{proof}
        因为
        \begin{equation*}
            \textstyle \mathrm{dim}\,\mathcal{R}(V^*) = \mathrm{dim}\,\Sigma^2(\bigwedge^2V^*) - \mathrm{dim}\,\bigwedge^4V^*
        \end{equation*}
        所以
        \begin{equation*}
            \mathrm{dim}\,\mathcal{R}(V^*) = \frac{1}{2}\binom{n}{2}\left(\binom{n}{2} - 1\right)-\binom{n}{4} = \frac{n^2(n^2-1)}{12}
        \end{equation*}
    \end{proof}

\subsection{相配二次型}

\subsection{曲率张量、挠率张量的坐标分量表示}
    设 $(M,g)$ 为黎曼流形, $\mathrm{Levi-Civita}$ 联络为 $\nabla$, $(U,\,x^i)$ 为一个局部坐标邻域,
    记 $g_{ij} = g\left(\pd{}{x^i},\pd{}{x^j}\right)$ 则:
    \begin{itemize}
        \item 挠率张量 $T(X,Y) = \lc{X}{Y}-\lc{Y}{X}-[X,Y]$
        \begin{equation*}
            T\big|_U = T_{ij}^{k}\pd{}{x^k}\otimes\md{x^i}\otimes\md{x^j}
        \end{equation*}
        其中
        \begin{align*}
            T_{ij}^{k} &= \md{x^k}\left(T\left(\pd{}{x^i},\pd{}{x^j}\right)\right) \\
            &= \md{x^k}\left(\lc{\pd{}{x^i}}{\pd{}{x^j}}-\lc{\pd{}{x^j}}{\pd{}{x^i}}\right) \\
            &= \Gamma_{ji}^{k}-\Gamma_{ij}^{k}
        \end{align*}
        于是
        \begin{equation*}
            T\big|_U = (\Gamma_{ji}^{k}-\Gamma_{ij}^{k})\pd{}{x^k}\otimes\md{x^i}\otimes\md{x^j}
        \end{equation*}
        \item 曲率张量 $\mathcal{R}(X,Y)Z = \lc{X}{\lc{Y}{Z}} - \lc{Y}{\lc{X}{Z}} - \lc{[X,Y]}{Z}$
        \begin{equation*}
            \mathcal{R}\big|_U = R_{ijk}^{l}\pd{}{x^l}\otimes\md{x^i}\otimes\md{x^j}\otimes\md{x^k}
        \end{equation*}
        其中
        \begin{align*}
            R_{ijk}^{l} &= \md{x^l}\left(\mathcal{R}\left(\pd{}{x^i},\pd{}{x^j}\right)\pd{}{x^k}\right) \\
            &= \md{x^l}\left(\lc{\pd{}{x^i}}{\lc{\pd{}{x^j}}{\pd{}{x^k}}} - \lc{\pd{}{x^j}}{\lc{\pd{}{x^i}}{\pd{}{x^k}}}\right) \\
            &= \md{x^l}\left(\lc{\pd{}{x^i}}{\left(\pd{}{x^p}\right)} - \lc{\pd{}{x^j}}{\left(\Gamma_{ki}^{p}\pd{}{x^p}\right)}\right) \\
            &= \md{x^l}\left(\pd{\Gamma_{kj}^{p}}{x^i}\pd{}{x^p} + \Gamma_{kj}^{p}\Gamma_{pi}^{q}\pd{}{x^q} - \pd{\Gamma_{ki}^{p}}{x^j}\pd{}{x^p} - \Gamma_{ki}^{p}\Gamma_{pj}^{q}\pd{}{x^q}\right) \\
            &= \pd{\Gamma_{kj}^{l}}{x^i} - \pd{\Gamma_{ki}^{l}}{x^j} + \Gamma_{kj}^{h}\Gamma_{hi}^{l} - \Gamma_{ki}^{h}\Gamma_{hj}^{l}
        \end{align*}
        因此
        \begin{equation*}
            \mathcal{R}\big|_U = \left(\pd{\Gamma_{kj}^{l}}{x^i} - \pd{\Gamma_{ki}^{l}}{x^j} + \Gamma_{kj}^{h}\Gamma_{hi}^{l} - \Gamma_{ki}^{h}\Gamma_{hj}^{l}\right)\pd{}{x^l}\otimes\md{x^i}\otimes\md{x^j}\otimes\md{x^k}
        \end{equation*}
        \item 黎曼曲率张量 $R(X,Y,Z,W) = \langle\mathcal{R}(X,Y)Z,W\rangle$
        \begin{equation*}
            R\big|_U = R_{ijkl}\md{x^i}\md{x^j}\md{x^k}\md{x^l}
        \end{equation*}
        其中
        \begin{equation*}
            R_{ijkl} = R_{ijk}^{h}g_{hl}
        \end{equation*}
        或者
        \begin{align*}
            & R_{ijkl} \\
            =& \left<\mathcal{R}\left(\pd{}{x^i},\pd{}{x^j}\right)\pd{}{x^k},\pd{}{x^l}\right> \\
            =& \left<\lc{\pd{}{x^i}}{\lc{\pd{}{x^j}}{\pd{}{x^k}}} - \lc{\pd{}{x^j}}{\lc{\pd{}{x^i}}{\pd{}{x^k}}},\pd{}{x^l}\right> \\
            =& \pd{}{x^i}\left<\lc{\pd{}{x^j}}{\pd{}{x^k}},\pd{}{x^l}\right> - \left<\lc{\pd{}{x^j}}{\pd{}{x^k}},\lc{\pd{}{x^i}}{\pd{}{x^l}}\right> \\
            & -\pd{}{x^j}\left<\lc{\pd{}{x^i}}{\pd{}{x^k}},\pd{}{x^l}\right> + \left<\lc{\pd{}{x^i}}{\pd{}{x^k}},\lc{\pd{}{x^j}}{\pd{}{x^l}}\right> \\
            =& \frac{1}{2}\pd{}{x^i}\left(\pd{}{x^j}\left<\pd{}{x^k},\pd{}{x^l}\right>+\pd{}{x^k}\left<\pd{}{x^j},\pd{}{x^l}\right>-\pd{}{x^l}\left<\pd{}{x^j},\pd{}{x^k}\right>\right) \\
            & -\frac{1}{2}\pd{}{x^j}\left(\pd{}{x^i}\left<\pd{}{x^k},\pd{}{x^l}\right>+\pd{}{x^k}\left<\pd{}{x^i},\pd{}{x^l}\right>-\pd{}{x^l}\left<\pd{}{x^i},\pd{}{x^k}\right>\right) \\
            & +\left<\lc{\pd{}{x^i}}{\pd{}{x^k}},\lc{\pd{}{x^j}}{\pd{}{x^l}}\right>-\left<\lc{\pd{}{x^j}}{\pd{}{x^k}},\lc{\pd{}{x^i}}{\pd{}{x^l}}\right> \\
            =& \frac{1}{2}\left(\npd{2}{g_{ik}}{x^{jl}}+\npd{2}{g_{jl}}{x^{ik}}-\npd{2}{g_{il}}{x^{jk}}-\npd{2}{g_{jk}}{x^{il}}\right) \\
            & +g_{pq}\Gamma_{ki}^{p}\Gamma_{lj}^{q}-g_{pq}\Gamma_{kj}^{p}\Gamma_{li}^{q}
        \end{align*}
        第四个等号用到了Koszul公式 \ref{eq:koszul}.
    \end{itemize}

\subsection{联络形式、挠率形式、曲率形式}
    除了用某个局部坐标给出的自然坐标标架场之外, 我们也经常用一般的标架场, 设 $\{e_i\}$ 为某个领域 $U$ 上处处线性无关的 $n$ 个向量场, $\{\omega^i\}$ 为其对偶一次微分式.
    \begin{remark}
        一般地, $\{e_i\}$ 不能定义在整个 $M$ 上, 因为这要求 $M$ 的切丛平凡.
    \end{remark}
    在这种一般的标架场下, 仍可以写出坐标分量表示, 设
    \begin{equation*}
        \lc{e_i}{e_j} = \Gamma_{ji}^{k}e_k
    \end{equation*}
    则
    \begin{equation*}
        \lc{}{e_j} = \Gamma_{ji}^{k}\omega^i\otimes e_k
    \end{equation*}
    \subsubsection{各个形式的定义}
        \begin{itemize}
            \item 联络形式:
            
            令一次外微分式
            \begin{equation*}
                \omega_{j}^{k} := \Gamma_{ji}^{k}\omega^i
            \end{equation*}
            于是
            \begin{equation*}
                \lc{}{e_j} = \omega_{j}^{k}\otimes e_k
            \end{equation*}
            我们称 $\{\omega_{j}^{k}\}$ 为黎曼流形的联络形式.
            \item 挠率形式:
            
            令二次外微分式
            \begin{equation*}
                \Omega^i := \md{\omega^i} + \omega^{i}_{j}\wedge\omega^j
            \end{equation*}
            可以验证
            \begin{equation*}
                \Omega^i = \frac{1}{2}T_{jk}^{i}\omega^j\wedge\omega^k
            \end{equation*}
            其中 $T_{jk}^{i}=\omega^i\left(T\left(e_j,e_k\right)\right)$, 我们称 $\{\Omega^i\}$ 为挠率形式, 此时
            \begin{equation*}
                T\big|_U = \Omega^i\otimes e_i
            \end{equation*}
            也即
            \begin{equation*}
                T(X,Y) = \Omega^i(X,Y)e_i
            \end{equation*}
            \item 曲率形式:
            
            令二次外微分式
            \begin{equation*}
                \Omega^{j}_{i} := \md\omega^{j}_{i} + \omega^{j}_{k}\wedge\omega^{k}_{i}
            \end{equation*}
            可以验证
            \begin{equation*}
                \Omega^{j}_{i} = \frac{1}{2}R_{ikl}^{j}\omega^k\omega^l
            \end{equation*}
            其中 $R_{ikl}^{j} = \omega^l\left(\mathcal{R}(e_i,e_k)e_l\right)$, 我们称 $\{\Omega^j_i\}$ 为曲率形式, 此时
            \begin{equation*}
                \mathcal{R}\big|_U = \Omega^j_i\otimes\omega^i\otimes e_j
            \end{equation*}
            也即
            \begin{equation*}
                \mathcal{R}(X,Y)e_i = \Omega^j_i(X,Y)e_j
            \end{equation*}
        \end{itemize}

        \begin{definition}[结构方程]
            我们称
            \begin{equation}
                \left\{
                \begin{aligned}
                    \md\omega^i = \Omega^i-\omega^i_j\wedge\omega^j \\
                    \md\omega^i_j = \Omega^i_j-\omega^i_k\wedge\omega^k_j
                \end{aligned}
                \right.
            \end{equation}
            为仿射联络空间 $(M,\nabla)$ 的 \textbf{结构方程}
        \end{definition}

        \begin{theorem}[第一、第二Bianchi恒等式]
            联络形式 $\omega^i_j$、挠率形式 $\Omega^i$ 和曲率形式 $\Omega^i_j$ 满足关系式
            \begin{equation} \label{eq:Bianchi1}
                \md\Omega^i = \omega^i_j\wedge\Omega^j-\Omega^i_j\wedge\omega^j
            \end{equation}
            \begin{equation} \label{eq:Bianchi2}
                \md\Omega^i_j = \omega^i_k\wedge\Omega^k_j-\Omega^i_k\wedge\omega^k_j
            \end{equation}
            其中式 (\ref{eq:Bianchi1}) 对应第一 \rm{Bianchi} 恒等式, 式 (\ref{eq:Bianchi2}) 对应第二 \rm{Bianchi} 恒等式.
        \end{theorem}
    \subsubsection{坐标变换下各个形式的变换}
        设 $\{e_i\}$ 和 $\{\tilde{e_i}\}$ 都是开子集 $U$ 上的切标架场, $\{\omega^i\}$ 和 $\{\tilde{\omega^i}\}$ 分别表示相应的余切标架场,
        $\{\omega^i_j\}$ 和 $\{\tilde{\omega^i_j}\}$ 表示相应的联络形式, 设坐标变换关系为
        \begin{equation*}
            \tilde{e_i} = a^j_ie_j\qquad \tilde{\omega^i} = b^i_j\omega^j
        \end{equation*}
        其中矩阵 $b = (b^i_j)$ 是 $a = (a^j_i)$ 的逆矩阵.
        则
        \begin{gather*}
            \tilde{\omega}^i_j = b^i_k\omega^l_j+b^i_k\omega^k_la^l_j \\
            \Omega^i = a^i_j\tilde{\Omega}^j \\
            \tilde{\Omega}^i_j = b^i_k\Omega^k_la^l_j
        \end{gather*}
        用矩阵表示即为
        \begin{equation*}
            \tilde{\omega} = a^{-1}\md a + a^{-1}\omega a \qquad \tilde{\Omega} = a^{-1}\Omega a
        \end{equation*}
    \subsubsection{用矩阵的语言重新表示联络形式}
        用分量表示以上微分式, 且用Einstein求和约定简写求和式总是不那么让人习惯, 这里用矩阵的语言重新封装一下, 能把一些复杂的计算看得更清楚.
        
        形式上我们可以写出以向量、微分形式为元素的矩阵, 并定义这种矩阵之间的乘法、张量积、外积以及求外微分. 为了快速进入主题, 我们省略了这些运算的定义、性质等内容, 
        下面直接进入正题:

        设 $e_1,\dots,e_n$ 为局部标架场, 
        \begin{equation}\label{eq:De_i}
            \lc{}{e_j} = \Gamma_{ji}^{k}e_k\otimes\omega^i = 
            (e_1,\cdots,e_n)
            \begin{pmatrix}
                \Gamma_{j1}^{1} & \cdots & \Gamma_{jn}^{1} \\
                \vdots & & \vdots \\
                \Gamma_{j1}^{n} & \cdots & \Gamma_{jn}^{n}
            \end{pmatrix}
            \begin{pmatrix}
                \omega^1 \\ \vdots \\ \omega^n
            \end{pmatrix}
        \end{equation}
        令
        \begin{equation*}
            \omega^i_j = 
            (\Gamma_{j1}^{i},\cdots,\Gamma_{jn}^{i})
            \begin{pmatrix}
                \omega_1 \\ \vdots \\ \omega_n
            \end{pmatrix}
        \end{equation*}
        代入式(\ref{eq:De_i})并打包整理可得到
        \begin{equation}\label{df:connection-form}
            \lc{}{(e_1,\cdots,e_n)} = 
            (e_1,\cdots,e_n)
            \begin{pmatrix}
                \omega^1_1 & \cdots & \omega^1_n \\
                \vdots & & \vdots \\
                \omega^n_1 & \cdots & \omega^n_n
            \end{pmatrix}
        \end{equation}
        记 $w := (\omega^1,\cdots,\omega^n)'$, $W := (\omega^i_j)$,  它们分别是余切标架场构成的列向量和联络形式构成的矩阵.

        定义挠率形式
        \begin{equation}\label{df:torsion-form}
            \begin{pmatrix}
                \Omega^1 \\ \vdots \\ \Omega^n
            \end{pmatrix} = 
            \md\begin{pmatrix}
                \omega^1 \\ \vdots \\ \omega^n
            \end{pmatrix} + 
            \begin{pmatrix}
                \omega^1_1 & \cdots & \omega^1_n \\
                \vdots & & \vdots \\
                \omega^n_1 & \cdots & \omega^n_n
            \end{pmatrix}\wedge
            \begin{pmatrix}
                \omega^1 \\ \vdots \\ \omega^n
            \end{pmatrix}
        \end{equation}
        定义曲率形式
        \begin{equation}\label{df:curvature-form}
            \begin{pmatrix}
                \Omega^1_1 & \cdots & \Omega^1_n \\ 
                \vdots & & \vdots \\ 
                \Omega^n_1 & \cdots & \Omega^n_n
            \end{pmatrix} = 
            \md\begin{pmatrix}
                \omega^1_1 & \cdots & \omega^1_n \\
                \vdots & & \vdots \\
                \omega^n_1 & \cdots & \omega^n_n
            \end{pmatrix} + 
            \begin{pmatrix}
                \omega^1_1 & \cdots & \omega^1_n \\
                \vdots & & \vdots \\
                \omega^n_1 & \cdots & \omega^n_n
            \end{pmatrix}\wedge
            \begin{pmatrix}
                \omega^1_1 & \cdots & \omega^1_n \\
                \vdots & & \vdots \\
                \omega^n_1 & \cdots & \omega^n_n
            \end{pmatrix}
        \end{equation}
        记 $\omega := (\Omega^1,\cdots,\Omega^n)'$, $\Omega := (\Omega^i_j)$, 则式(\ref{df:torsion-form})和(\ref{df:curvature-form})可写为
        \begin{numcases}{}
            \omega = \md w + W\wedge w \label{df:matrix-torsion-form}\\
            \Omega = \md W + W\wedge W \label{df:matrix-curvature-form}
        \end{numcases}
        \begin{remark}
            注意挠率形式向量 $\omega$ 和余切标架向量 $w$ 长得很像, 暂时没想到更好的记号, 注意区分.
        \end{remark}
        \begin{proposition}[第一、第二Bianchi恒等式]
            \begin{numcases}{}
            \md\omega = \Omega\wedge w - W\wedge w \label{eq:matrix-Bianchi1}\\
            \md\Omega = \Omega\wedge W - W\wedge\Omega \label{eq:matrix-Bianchi2}
        \end{numcases}
        \end{proposition}
        \begin{proof}
            第一Bianchi恒等式:
            \begin{align*}
                \md\omega &= \md(\md w + W\wedge w) = \md W\wedge w - W\wedge\md w \\
                &= (\Omega-W\wedge W)\wedge w - W\wedge(\omega-W\wedge w) \\
                &= \Omega\wedge w - W\wedge\omega 
            \end{align*}
            第二Bianchi恒等式:
            \begin{align*}
                \md\Omega &= \md(\md W + W\wedge W) = \md W\wedge W - W\wedge\md W \\
                &= (\Omega - W\wedge W)\wedge W - W\wedge(\Omega - W\wedge W) \\
                &= \Omega\wedge W - W\wedge\Omega
            \end{align*}
        \end{proof}
        \begin{proposition}[坐标变换]
            设 $\{e_i\}$、$\{\tilde{e_i}\}$ 是两个局部标架场, $\{\omega^i\}$、$\{\tilde{\omega^i}\}$ 分别对应它们的对偶1-形式, 它们之间的坐标变换为
            \begin{equation*}
                (e_1,\cdots,e_n) = (\tilde{e}_1,\cdots,\tilde{e}_n)
                \begin{pmatrix}
                    a^1_1 & \cdots & a^1_n \\
                    \vdots & & \vdots \\
                    a^n_1 & \cdots & a^n_n
                \end{pmatrix}
            \end{equation*}
            则对偶1-形式之间的变换为
            \begin{equation*}
                \begin{pmatrix}
                    \omega^1 \\ \vdots \\ \omega^n
                \end{pmatrix} = 
                \begin{pmatrix}
                    a^1_1 & \cdots & a^1_n \\
                    \vdots & & \vdots \\
                    a^n_1 & \cdots & a^n_n
                \end{pmatrix}^{-1}
                \begin{pmatrix}
                    \tilde{\omega}^1 \\ \vdots \\ \tilde{\omega}^n
                \end{pmatrix}
            \end{equation*}
            记过渡矩阵 $a := (a^i_j)$, 则上述式子可改写为
            \begin{equation}
                (e_1,\cdots,e_n) = (\tilde{e}_1,\cdots,\tilde{e}_n)a
            \end{equation}
            和
            \begin{equation}
                \begin{pmatrix}
                    \omega^1 \\ \vdots \\ \omega^n
                \end{pmatrix} = a^{-1}
                \begin{pmatrix}
                    \tilde{\omega}^1 \\ \vdots \\ \tilde{\omega}^n
                \end{pmatrix}
            \end{equation}
            此时联络形式之间的变换为
            \begin{equation}\label{eq:Transformation-connection-form}
                \tilde{w} = a^{-1}\md a + a^{-1}wa
            \end{equation}
            挠率形式之间的变换为
            \begin{equation}\label{eq:Transformation-torsion-form}
                \tilde{\omega} = a^{-1}\omega
            \end{equation}
            曲率形式之间的变换为
            \begin{equation}\label{eq:Transformation-curvature-form}
                \tilde{\Omega} = a^{-1}\Omega a
            \end{equation}
        \end{proposition}
        \begin{proof}
            对于式(\ref{eq:Transformation-connection-form}), 因为
            \begin{align*}
                \lc{}{(\tilde{e}_1,\cdots,\tilde{e}_n)} =& \lc{}{\left((e_1,\cdots,e_n)a\right)} \\
                =& \left(\lc{}{(e_1,\cdots,e_n)}\right)a + (e_1,\cdots,e_n)\lc{}{a} \\
                =& (e_1,\cdots,e_n)wa + (e_1,\cdots,e_n)\md a \\
                =& (\tilde{e}_1,\cdots,\tilde{e}_n)a^{-1}(wa + \md a) \\
                =& (\tilde{e}_1,\cdots,\tilde{e}_n)\tilde{w}
            \end{align*}
            故
            \begin{equation*}
                \tilde{w} = a^{-1}wa + a^{-1}\md a
            \end{equation*}
            对于式(\ref{eq:Transformation-torsion-form}), 因为
            \begin{align*}
                \tilde{\omega} =& \md\tilde{w} + \tilde{W}\wedge\tilde{w} \\
                =& \md(a^{-1}w) + (a^{-1}\md a + a^{-1}Wa)\wedge a^{-1}w \\
                =& \md a^{-1}w + a^{-1}\md w + a^{-1}(\md a)a^{-1}\wedge w + a^{-1}W\wedge w
            \end{align*}
            由
            \begin{equation*}
                0 = \md(a^{-1}a) = (\md a^{-1})a + a^{-1}\md a
            \end{equation*}
            所以
            \begin{equation}\label{eq:da^-1}
                \md a^{-1} = -a^{-1}(\md a)a^{-1}
            \end{equation}
            代入可得
            \begin{equation*}
                \tilde{\omega} = a^{-1}\md w + a^{-1}W\wedge w = a^{-1}\omega
            \end{equation*}
            对于式(\ref{eq:Transformation-curvature-form}), 因为
            \begin{align*}
                \tilde{\Omega} =& \md\tilde{W} + \tilde{W}\wedge\tilde{W} \\
                =& \md(a^{-1}\md a + a^{-1}Wa) + (a^{-1}\md a + a^{-1}Wa)\wedge(a^{-1}\md a + a^{-1}Wa) \\
                =& \md a^{-1}\wedge\md a + \md a^{-1}\wedge Wa + a^{-1}\md Wa - a^{-1}W\wedge\md a \\
                &+ a^{-1}(\md a)a^{-1}\wedge\md a + a^{-1}(\md a)a^{-1}\wedge Wa + a^{-1}W\wedge\md a + a^{-1}W\wedge Wa \\
                =& a^{-1}(\md W + W\wedge W)a
            \end{align*}
            其中第四个等式用到了式(\ref{eq:da^-1}), 所以
            \begin{equation*}
                \tilde{\Omega} = a^{-1}\Omega a
            \end{equation*}
        \end{proof}
