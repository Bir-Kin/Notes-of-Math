\chapter{微分流形}
    \section{向量丛结构群的约化}
    \begin{definition}[向量丛的定义]
        设 $E$, $M$ 为微分流形, $\pi:E\rightarrow M$ 为光滑满射, 且有 $M$ 的开覆盖 $\{U_{\alpha}\}$ 
        及微分同胚 $\psi_{\alpha}:\pi^{-1}(U_{\alpha})\rightarrow U_{\alpha}\times\mathbb{R}^k$, 满足:
        \begin{enumerate}
            \item\label{def:bundle1} $\psi(\pi^{-1}(p))=\{p\}\times\mathbb{R}^k,\;\forall p\in U_{\alpha}$,
            \item\label{def:bundle2} 当 $U_{\alpha}\cap U_{\beta}\neq\emptyset$ 时, 存在光滑映射 $g_{\alpha\beta}:U_{\alpha}\cap U_{\beta}\rightarrow{\rm GL}(k,\mathbb{R})$, 使得 $\psi_{\beta}\circ\psi_{\alpha}^{-1}(p,\,v)=(p,\,g_{\beta\alpha}(p)v)$.
        \end{enumerate}
        则称:
        \begin{itemize}
            \item $E$ 是 $M$ 上的光滑向量丛, $k$ 为向量丛的秩, $\pi$ 为丛投影;
            \item $\left\{(U_{\alpha}, \psi_{\alpha})\right\}$ 为局部平凡化, $g_{\beta\alpha}$ 为连接函数, ${\rm GL}(k,\mathbb{R})$ 为结构群;
            \item $E_p:=\pi^{-1}(p)$ 为点 $p$ 上的纤维.
        \end{itemize}
        对每个 $E_p$, 由条件\ref{def:bundle1}可知 $E_p$ 上可自然定义一个线性空间结构, 这看似依赖于局部平凡化 $\psi_{\alpha}$ 的选取, 不过由条件\ref{def:bundle2}可知线性结构并不依赖局部平凡化的选取.
        
        若存在 ${\rm GL}(k,\mathbb{R})$ 的闭 {\rm Lie} 子群 $H$, 使得 $g_{\beta\alpha}(p)\in H,\;\forall\,p\in U_{\alpha}\cap U_{\beta}$, 则称结构群{\bf 可约化到子群 $H$}.
    \end{definition}

    连接函数 $g_{\beta\alpha}$ 在向量丛的定义中占据很重要的地位, 容易证明它满足性质:
    \begin{equation*}
        g_{\alpha\alpha} = 1,\;\forall\,U_{\alpha},\qquad g_{\alpha\beta}g_{\beta\gamma}g_{\gamma\alpha} = 1,\;\forall\,U_{\alpha}\cap U_{\beta}\cap U_{\gamma}\neq\emptyset.
    \end{equation*}
    反之, 若有一族光滑函数 $\{g_{\alpha\beta}\}$ 满足以上性质, 定义商空间 $E:=\sqcup_{\alpha}(U_{\alpha}\times\mathbb{R}^k)\big/\sim$, 其中等价关系定义为: $(p,v_{\alpha})\in U_{\alpha}\times\mathbb{R}^k,\;(q,v_{\beta})\in U_{\beta}\times\mathbb{R}^k$
    \begin{equation*}
        (p,v_{\alpha})\sim(q,v_{\beta})\Leftrightarrow p=q,\;v_{\beta}=g_{\beta\alpha}(p)v_{\alpha}.
    \end{equation*}
    $E$ 的拓扑由商拓扑给出, 记 $[p,v]$ 为 $(p,v)$ 的等价类, 定义 $\pi:E\rightarrow M$, $\pi([p,v])=p$. 则 $E$ 在投影映射 $\pi$ 下成为 $M$ 上的秩 $k$ 的向量丛. 

    \subsection{流形可定向与结构群可约化至 ${\rm GL^+(k,\mathbb{R})}$}
    略

    \subsection{黎曼度量与结构群可约化至正交群 $O(k)$}
    流形 $M$ 上的黎曼度量是指光滑 $(0,2)$-张量场 $g$, $g$ 在每个点的切空间处都是内积.
    下面就来说明 $n$ 维流形 $M$ 上存在黎曼结构与切丛 $TM$ 的结构群可约化至正交群 $O(n)$ 是等价的.

    $1^{\circ}.$ 设 $(M,g)$ 为一个黎曼流形, 取 $M$ 的一个局部坐标覆盖 $\{(U_{\alpha};x^1_{\alpha},\dots,x^n_{\alpha})\}$, 于是 $\pd{}{x^1_{\alpha}},\dots,\pd{}{x^n_{\alpha}}$ 成为 $U_{\alpha}$ 上的一组标架, 因为 $U_{\alpha}$ 上有度量结构, 
    我们可对标架做 Gram-Schmidt 正交化得到单位正交标架 $e_{1\alpha},\dots,e_{n\alpha}$, 令局部平凡化映射为
    \begin{align*}
        \psi_{\alpha}:TU_{\alpha}&\rightarrow U_{\alpha}\times\mathbb{R}^n \\
        (p,a^ie_{i\alpha}|_p)&\mapsto(p,a^ie_i)
    \end{align*}
    其中 $e_1,\dots,e_n$ 表示 $\mathbb{R}^n$ 上的自然基底. 当 $U_{\alpha}\cap U_{\beta}\neq\emptyset$ 时, 对每个点 $p\in U_{\alpha}\cap U_{\beta}$, 因为 $\{e_{i\alpha}|_p\}$ 和 $\{e_{i\beta}|_p\}$ 都是 $T_pM$ 的一组标准正交基, 
    所以转移函数 $g_{\beta\alpha}(p)$ 是正交矩阵, 因此结构群可被约化至 $O(n)$.

    $2^{\circ}.$ 假设 $TM$ 的结构群可约化至正交群, 设 $\{(U_{\alpha},\psi_{\alpha})\}$ 是对应的平凡化, 即 $\psi_{\alpha}$ 是从 $TU_{\alpha}$ 到 $U_{\alpha}\times\mathbb{R}^n$ 的微分同胚, 
    令 $e_{i\alpha} = \psi^{-1}(U_{\alpha}\times \{e_i\})$, 其中 $\{e_i\}$ 为 $\mathbb{R}^n$ 的自然基底. 我们得到了 $TU_{\alpha}$ 上处处线性无关的一组向量场 $\{e_{i\alpha}\}$, 命这组向量场构成 $TU_{\alpha}$ 的一个单位正交标架场, 这能唯一确定 $TU_{\alpha}$ 上的黎曼度量.
    若 $U_{\alpha}\cap U_{\beta}\neq\emptyset$, 对 $\forall\,p\in U_{\alpha}\cap U_{\beta}$, 
    \begin{align*}
        \langle e_{i\alpha},e_{j\alpha}\rangle_p &= \langle\psi_{\alpha}(e_{i\alpha}|_p),\psi_{\alpha}(e_{j\alpha}|_p)\rangle \\
        &= \langle g_{\alpha\beta}(p)\psi_{\beta}(e_{i\beta}|_p),g_{\alpha\beta}(p)\psi_{\beta}(e_{j\beta}|_p)\rangle \\
        &= \langle\psi_{\beta}(e_{i\beta}|_p),\psi_{\beta}(e_{j\beta}|_p)\rangle \\
        &= \langle e_{i\beta},e_{j\beta}\rangle_p
    \end{align*}
    所以不同平凡化定义的黎曼结构是相容的, 因此能定义一个整体的黎曼度量 $g$.

    注意到我们能用单位分解在任意微分流形上构造黎曼度量, 这表明任意微分流形切丛的结构群都能约化到正交群.

    \subsection{复向量丛与近复结构, 与结构群可约化至 ${\rm GL}(k,\mathbb{C})$}
    设 $M$ 是 $m$ 维流形, $M$ 上的复向量丛 $E$ 在定义上仅需要把纤维 $\mathbb{R}^k$ 改为 $\mathbb{C}^k$、结构群改为 ${\rm GL}(k,\mathbb{C})$. 

    但如果把 $\mathbb{C}^k$ 视为 $\mathbb{R}^{2k}$, 则结构群可约化至 ${\rm GL}(2k,\mathbb{R})$ 的子群 
    \begin{equation*}
        \left\{\begin{pmatrix}
            A & -B \\
            B & A
        \end{pmatrix}\,\bigg|\,|A|^2+|B|^2>0\right\}
    \end{equation*}
    我们仍把这个子群记为 ${\rm GL}(k,\mathbb{C})$. 可以证明实的秩为 $2k$ 的向量丛 $E$ 为复的秩为 $k$ 的向量丛当且仅当结构群可约化至 ${\rm GL}(k,\mathbb{C})$.

    我们也可以从近复结构的视角理解复向量丛, 若实的秩为 $2k$ 的向量丛 $E$ 上存在自同构 $J$ (即 $\pi\circ J = \pi$), 使得 $J^2 = -{\rm id}$, 则称 $J$ 为 $M$ 的近复结构. 可以证明 $M$ 为复向量丛当且仅当 $M$ 上存在近复结构.
    
    一方面若 $M$ 为复向量丛, 则可以逐点定义 $J_p(p,v)=(p,\sqrt{-1}v)$, 因为转移映射是复线性变换, 所以 $J_p$ 良定, 且 $J_p^2 = -\id$; 另一方面我们可以适当修改平凡化 $\psi_{\alpha}$ 使得 $J$ 可局部表示为 
    \begin{equation*}
        J_{\alpha}(p,v_{\alpha}) = \left(p,
        \begin{pmatrix}
            & -I_k \\
            I_k & 
        \end{pmatrix}v_{\alpha}\right)
    \end{equation*}
    因为 $g_{\alpha\beta}\cdot J_{\beta} = J_{\alpha}\cdot g_{\alpha\beta}$, 所以
    \begin{equation*}
        \begin{pmatrix}
            & -I_k \\
            I_k & 
        \end{pmatrix}\cdot g_{\alpha\beta}(p) = g_{\alpha\beta}(p)\cdot
        \begin{pmatrix}
            & -I_k \\
            I_k & 
        \end{pmatrix}\Rightarrow g_{\alpha\beta}(p) = 
        \begin{pmatrix}
            A & -B \\
            B & A
        \end{pmatrix}
    \end{equation*}
    从而结构群可约化至 $\GL(k,\mC)$.

    \section{向量丛分类定理}
    \subsection{同伦的映射拉回同构的向量丛(纤维丛)}
    \begin{definition}[拉回丛的定义]
        设 $f:X\rightarrow Y$, 且有向量丛 $p:E\rightarrow Y$, 则可以定义 $X$ 上的拉回丛 $p':f^*E\rightarrow X$, 其中 
        \begin{equation*}
            f^*E:=\{(x,e)\in X\times E\,\big|\,f(x)=p(e)\}
        \end{equation*}
        为 $X\times E$ 的子集, 且赋予子拓扑结构. 丛投影为映射到第一个分量的投影映射. 每根纤维的线性结构由 $E$ 上每根纤维的线性结构给出.(有模糊的地方)
    \end{definition}
    \begin{proposition}[同伦的映射拉回同构的向量丛]
        现有向量丛 $p:E\rightarrow Y$, 设 $f\simeq g:X\rightarrow Y$ 为同伦的光滑映射, 则拉回丛 $f^*E$ 与 $g^*E$ 丛同构.
    \end{proposition}

    在证明之前, 我们先分析一下命题. 设 $H:X\times[0,1]\rightarrow Y$ 是从 $f$ 到 $g$ 的光滑伦移, 即 $H|_{X\times\{0\}}=f$, $H|_{X\times\{1\}}=g$. 则有 $X\times[0,1]$ 上的拉回丛 $H^*E$, 
    且 $H^*E|_{X\times\{0\}}=f^*E,\,H^*E|_{X\times\{1\}}=g^*E$. 因此为了证明 $f^*E\cong g^*E$, 只需证明:

    \begin{proposition}[向量丛在柱空间的上下底的限制是同构的]
        当 $X$ 仿紧时, 对任意 $X\times[0,1]$ 上的向量丛 $E$, $E|_{X\times\{0\}}\cong E|_{X\times\{1\}}$.
    \end{proposition}
    \begin{proof}
        我们需要两个关于向量丛的事实:

        $(1):$ 若 $p:E\rightarrow X\times[a,b]$ 在 $X\times[a,c]$ 和 $X\times[c,b]$ 上分别是平凡的, 则 $E$ 在整个 $X\times[a,b]$ 上平凡. 
        
        只需分别写出在 $X\times[a,c]$ 和 $X\times[c,b]$ 
        上的平凡化 $h_1$ 和 $h_2$, 并修改 $h_2$ 使得它们在 $p^{-1}(X\times\{c\})$ 上匹配, 则 $h_1$ 和修改后的 $h_2$ 合并成整个 $X\times[a,b]$ 上的平凡化.

        $(2):$ 对于向量丛 $p:E\rightarrow X\times[0,1]$, 存在 $X$ 的开覆盖 $\{U_{\alpha}\}$ 使得 $E$ 在每个 $U_{\alpha}\times[0,1]$ 上都是平凡的. 
        
        对任意 $x\in X$, 存在 $U_{x,1},\dots,U_{x,k}$ 
        以及 $0=t_0<t_1<\cdots<t_k=1$ 使得 $E$ 在 $U_{x,i}\times[t_{i-1},t_i]$ 上平凡, 令 $U_x=U_{x,1}\cap\cdots\cap U_{x,k}$, 则由 $(1)$ 知 $E$ 在 $U_x\times[0,1]$ 上平凡.

        下面我们来证明该命题, 
    \end{proof}

    \section{Kunn\"{e}th 公式与 Leray-Hirsch 定理}
    \begin{theorem}[Kunn\"{e}th 公式]
        设流形 $M$ 有有限好覆盖, $F$ 是任意流形, 则
        \begin{equation*}
            H^*(M\times F)\cong H^*(M)\otimes H^*(F)
        \end{equation*}
    \end{theorem}

    \begin{proof}[证明概要]
        设 $\pi:M\times F\rightarrow M$, $\rho:M\times F\rightarrow F$ 为乘积流形到两个分量的投影, 则可以定义
        \begin{align*}
            \psi:H^*(M)\otimes H^*(F)&\rightarrow H^*(M\times F) \\
            \omega\otimes\tau&\mapsto\pi^*\omega\wedge\rho^*\tau
        \end{align*}
        由 M-V 论证, 可以得到如下交换图:

        \begin{tikzcd}[column sep = tiny]
            \vdots \arrow[d]                                                                                                 & \vdots \arrow[d]                                          \\
            \bigoplus\limits_{p+q=k}H^p(U\cup V)\otimes H^q(F) \arrow[d] \arrow[r, "\psi"]                                   & H^k\left((U\cup V)\times F\right) \arrow[d]               \\
            \bigoplus\limits_{p+q=k}\left((H^p(U)\otimes H^q(F))\oplus (H^p(V)\otimes H^q(F))\right) \arrow[r, "\psi"] \arrow[d] &             H^k(U\times F)\oplus H^k(V\times F) \arrow[d] \\
            \bigoplus\limits_{p+q=k}H^p(U\cap V)\otimes H^q(F) \arrow[r, "\psi"] \arrow[d, "\md^*"]                          & H^k((U\cap V)\times F) \arrow[d, "\md^*"]                 \\
            \bigoplus\limits_{p+q=k+1}H^p(U\cup V)\otimes H^q(F) \arrow[d] \arrow[r, "\psi"]                                 & H^{k+1}\left((U\cup V)\times F\right) \arrow[d]           \\
            \vdots                                                                                                           & \vdots                                                   
        \end{tikzcd}

        前两个圈的交换性显然, 第三个圈的交换性需要用到 $\md^*$ 的表达式. 由五引理能得到归纳递推, 归纳奠基是平凡的.
    \end{proof}

    \begin{theorem}[Leray-Hirsch 定理]
        设 $\pi:E\rightarrow M$ 为纤维丛, 纤维为 $F$, 若存在 $E$ 上的微分形式 $\{\ft{e}{1}{n}\}$ 满足将它们限制在每个纤维 $F_x$ 上都能得到 $H^*(F_x)$ 的一组基, 
        则
        \begin{equation*}
            H^*(E)\cong H^*(M)\otimes\{\ft{e}{1}{n}\}\cong H^*(M)\otimes H^*(F).
        \end{equation*}
    \end{theorem}
    \begin{proof}[证明概要]
        这里的关键在于不存在 $E$ 到 $F$ 的整体投影 $\rho$, 也就无法通过这个方式定义 $\rho^*:H^*(F)\rightarrow H^*(E)$ 了. 
        但是借助 $\{\ft{e}{1}{n}\}$ 我们可以构造合适的映射 $\tilde{\rho^*}$, 做法如下:固定某个点 $x\in M$, 也即固定某个纤维 $F_x$, 
        取 $H^*(F_x)$ 的一组基 $\{\ft{f}{1}{n}\}$, 定义:
        \begin{align*}
            \tilde{\rho^*}:H^*(F)&\rightarrow H^*(E) \\
            \sum_{i}a_if_i&\mapsto\sum_{i}a_ie_i.
        \end{align*} 

        于是可以定义:
        \begin{align*}
            \tilde{\psi}:H^*(M)\otimes H^*(F)&\rightarrow H^*(E) \\
            \omega\otimes\tau&\mapsto\pi^*\omega\wedge\tilde{\rho^*}\tau
        \end{align*}
        归纳递推仍由 M-V 论证给出; 
        
        \begin{tikzcd}[column sep = tiny]
            \vdots \arrow[d]                                                                                                 & \vdots \arrow[d]                                          \\
            \bigoplus\limits_{p+q=k}H^p(U\cup V)\otimes H^q(F) \arrow[d] \arrow[r, "\psi"]                                   & H^k\left(\pi^{-1}(U\cup V)\right) \arrow[d]               \\
            \bigoplus\limits_{p+q=k}\left((H^p(U)\otimes H^q(F))\oplus (H^p(V)\otimes H^q(F))\right) \arrow[r, "\psi"] \arrow[d] &             H^k(\pi^{-1}(U))\oplus H^k(\pi^{-1}(V)) \arrow[d] \\
            \bigoplus\limits_{p+q=k}H^p(U\cap V)\otimes H^q(F) \arrow[r, "\psi"] \arrow[d, "\md^*"]                          & H^k(\pi^{-1}(U\cap V)) \arrow[d, "\md^*"]                 \\
            \bigoplus\limits_{p+q=k+1}H^p(U\cup V)\otimes H^q(F) \arrow[d] \arrow[r, "\psi"]                                 & H^{k+1}\left(\pi^{-1}(U\cup V)\right) \arrow[d]           \\
            \vdots                                                                                                           & \vdots                                                   
        \end{tikzcd}
        
        因为 good cover 中的每个开集都同伦于单点, 而同伦映射诱导同构的拉回丛(注意这里是纤维丛的版本), 因此 good cover 同时也是 locally trivialization.
        故 $\pi^{-1}(U_\alpha)\cong U_{\alpha}\times F$, 从而 $H^*(\pi^{-1}(U_\alpha))\cong H^*(U_{\alpha})\otimes H^*(F)$, 这给出了归纳奠基.
    \end{proof}
    \begin{remark}
        实际上当底空间 $M$ 连通时, 定理的条件可弱化为 $\{\ft{e}{1}{n}\}$ 限制在某个纤维 $F_x$ 上得到 $H^*(F_x)$ 的一组基. 
        因为对不同的两点 $x,y$, 有道路 $\gamma:[0,1]\rightarrow M$ 将他们相连, 于是嵌入映射 $\iota_x:F_x\hookrightarrow E$ 和 $\iota_y:F_y\hookrightarrow E$ 同伦.
        因此拉回映射 $\iota^*_x:H^*(E)\rightarrow H^*(F_x)$ 与 $\iota^*_y:H^*(E)\rightarrow H^*(F_y)$ 相等.
    \end{remark}
    \begin{remark}
        这里的同构 $H^*(E)\cong H^*(M)\otimes H^*(F)$ 并不保持环结构(例如?)因此只能说 $H^*(E)$ 可看成 $H^*(M)$-模.
    \end{remark}
    \begin{remark}
        存在不满足{\rm Leray-Hirsch}定理条件的纤维丛, 比如{\rm Hopf}纤维化:
        \begin{center}
            \begin{tikzcd}
                S^1 \arrow[r] & S^3 \arrow[d] \\
                              & S^2          
            \end{tikzcd}    
        \end{center}
        其中
        \begin{equation*}
            H^*(S^3)\neq H^*(S^1)\otimes H^*(S^2)
        \end{equation*}
    \end{remark}