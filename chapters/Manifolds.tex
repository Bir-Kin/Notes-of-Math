\chapter{微分流形}
    \section{向量丛结构群的约化}
    \begin{definition}[向量丛的定义]
        设 $E$, $M$ 为微分流形, $\pi:E\rightarrow M$ 为光滑满射, 且有 $M$ 的开覆盖 $\{U_{\alpha}\}$ 
        及微分同胚 $\psi_{\alpha}:\pi^{-1}(U_{\alpha})\rightarrow U_{\alpha}\times\mathbb{R}^k$, 满足:
        \begin{enumerate}
            \item\label{def:bundle1} $\psi(\pi^{-1}(p))=\{p\}\times\mathbb{R}^k,\;\forall p\in U_{\alpha}$,
            \item\label{def:bundle2} 当 $U_{\alpha}\cap U_{\beta}\neq\emptyset$ 时, 存在光滑映射 $g_{\alpha\beta}:U_{\alpha}\cap U_{\beta}\rightarrow{\rm GL}(k,\mathbb{R})$, 使得 $\psi_{\beta}\circ\psi_{\alpha}^{-1}(p,\,v)=(p,\,g_{\beta\alpha}(p)v)$.
        \end{enumerate}
        则称:
        \begin{itemize}
            \item $E$ 是 $M$ 上的光滑向量丛, $k$ 为向量丛的秩, $\pi$ 为丛投影;
            \item $\left\{(U_{\alpha}, \psi_{\alpha})\right\}$ 为局部平凡化, $g_{\beta\alpha}$ 为连接函数, ${\rm GL}(k,\mathbb{R})$ 为结构群;
            \item $E_p:=\pi^{-1}(p)$ 为点 $p$ 上的纤维.
        \end{itemize}
        对每个 $E_p$, 由条件\ref{def:bundle1}可知 $E_p$ 上可自然定义一个线性空间结构, 这看似依赖于局部平凡化 $\psi_{\alpha}$ 的选取, 不过由条件\ref{def:bundle2}可知线性结构并不依赖局部平凡化的选取.
        
        若存在 ${\rm GL}(k,\mathbb{R})$ 的闭 {\rm Lie} 子群 $H$, 使得 $g_{\beta\alpha}(p)\in H,\;\forall\,p\in U_{\alpha}\cap U_{\beta}$, 则称结构群{\bf 可约化到子群 $H$}.
    \end{definition}

    连接函数 $g_{\beta\alpha}$ 在向量丛的定义中占据很重要的地位, 容易证明它满足性质:
    \begin{equation*}
        g_{\alpha\alpha} = 1,\;\forall\,U_{\alpha},\qquad g_{\alpha\beta}g_{\beta\gamma}g_{\gamma\alpha} = 1,\;\forall\,U_{\alpha}\cap U_{\beta}\cap U_{\gamma}\neq\emptyset.
    \end{equation*}
    反之, 若有一族光滑函数 $\{g_{\alpha\beta}\}$ 满足以上性质, 定义商空间 $E:=\sqcup_{\alpha}(U_{\alpha}\times\mathbb{R}^k)\big/\sim$, 其中等价关系定义为: $(p,v_{\alpha})\in U_{\alpha}\times\mathbb{R}^k,\;(q,v_{\beta})\in U_{\beta}\times\mathbb{R}^k$
    \begin{equation*}
        (p,v_{\alpha})\sim(q,v_{\beta})\Leftrightarrow p=q,\;v_{\beta}=g_{\beta\alpha}(p)v_{\alpha}.
    \end{equation*}
    $E$ 的拓扑由商拓扑给出, 记 $[p,v]$ 为 $(p,v)$ 的等价类, 定义 $\pi:E\rightarrow M$, $\pi([p,v])=p$. 则 $E$ 在投影映射 $\pi$ 下成为 $M$ 上的秩 $k$ 的向量丛.

    \subsection{流形可定向与结构群可约化至 ${\rm GL^+(k,\mathbb{R})}$}

    \subsection{黎曼度量与结构群可约化至正交群 $O(k)$}
    流形 $M$ 上的黎曼度量是指光滑 $(0,2)$-张量场 $g$, $g$ 在每个点的切空间处都是内积.
    下面就来说明 $n$ 维流形 $M$ 上存在黎曼结构与切丛 $TM$ 的结构群可约化至正交群 $O(n)$ 是等价的.

    $1^{\circ}.$ 设 $(M,g)$ 为一个黎曼流形, 取 $M$ 的一个局部坐标覆盖 $\{(U_{\alpha};x^1_{\alpha},\dots,x^n_{\alpha})\}$, 于是 $\pd{}{x^1_{\alpha}},\dots,\pd{}{x^n_{\alpha}}$ 成为 $U_{\alpha}$ 上的一组标架, 因为 $U_{\alpha}$ 上有度量结构, 
    我们可对标架做 Gram-Schmidt 正交化得到单位正交标架 $e_{1\alpha},\dots,e_{n\alpha}$, 令局部平凡化映射为
    \begin{align*}
        \psi_{\alpha}:TU_{\alpha}&\rightarrow U_{\alpha}\times\mathbb{R}^n \\
        (p,a^ie_{i\alpha}|_p)&\mapsto(p,a^ie_i)
    \end{align*}
    其中 $e_1,\dots,e_n$ 表示 $\mathbb{R}^n$ 上的自然基底. 当 $U_{\alpha}\cap U_{\beta}\neq\emptyset$ 时, 对每个点 $p\in U_{\alpha}\cap U_{\beta}$, 因为 $\{e_{i\alpha}|_p\}$ 和 $\{e_{i\beta}|_p\}$ 都是 $T_pM$ 的一组标准正交基, 
    所以转移函数 $g_{\beta\alpha}(p)$ 是正交矩阵, 因此结构群可被约化至 $O(n)$.

    $2^{\circ}.$ 假设 $TM$ 的结构群可约化至正交群, 设 $\{(U_{\alpha},\psi_{\alpha})\}$ 是对应的平凡化, 即 $\psi_{\alpha}$ 是从 $TU_{\alpha}$ 到 $U_{\alpha}\times\mathbb{R}^n$ 的微分同胚, 
    令 $e_{i\alpha} = \psi^{-1}(U_{\alpha}\times \{e_i\})$, 其中 $\{e_i\}$ 为 $\mathbb{R}^n$ 的自然基底. 我们得到了 $TU_{\alpha}$ 上处处线性无关的一组向量场 $\{e_{i\alpha}\}$, 命这组向量场构成 $TU_{\alpha}$ 的一个单位正交标架场, 这能唯一确定 $TU_{\alpha}$ 上的黎曼度量.
    若 $U_{\alpha}\cap U_{\beta}\neq\emptyset$, 对 $\forall\,p\in U_{\alpha}\cap U_{\beta}$, 
    \begin{align*}
        \langle e_{i\alpha},e_{j\alpha}\rangle_p &= \langle\psi_{\alpha}(e_{i\alpha}|_p),\psi_{\alpha}(e_{j\alpha}|_p)\rangle \\
        &= \langle g_{\alpha\beta}(p)\psi_{\beta}(e_{i\beta}|_p),g_{\alpha\beta}(p)\psi_{\beta}(e_{j\beta}|_p)\rangle \\
        &= \langle\psi_{\beta}(e_{i\beta}|_p),\psi_{\beta}(e_{j\beta}|_p)\rangle \\
        &= \langle e_{i\beta},e_{j\beta}\rangle_p
    \end{align*}
    所以不同平凡化定义的黎曼结构是相容的, 因此能定义一个整体的黎曼度量 $g$.

    注意到我们能用单位分解在任意微分流形上构造黎曼度量, 这表明任意微分流形切丛的结构群都能约化到正交群.

    \section{向量丛分类定理}
    \subsection{同伦的映射拉回同构的向量丛(纤维丛)}
    \begin{definition}[拉回丛的定义]
        设 $f:X\rightarrow Y$, 且有向量丛 $p:E\rightarrow Y$, 则可以定义 $X$ 上的拉回丛 $p':f^*E\rightarrow X$, 其中 
        \begin{equation*}
            f^*E:=\{(x,e)\in X\times E\,\big|\,f(x)=p(e)\}
        \end{equation*}
        为 $X\times E$ 的子集, 且赋予子拓扑结构. 丛投影为映射到第一个分量的投影映射. 每根纤维的线性结构由 $E$ 上每根纤维的线性结构给出.(有模糊的地方)
    \end{definition}
    \begin{proposition}[同伦的映射拉回同构的向量丛]
        现有向量丛 $p:E\rightarrow Y$, 设 $f\simeq g:X\rightarrow Y$ 为同伦的光滑映射, 则拉回丛 $f^*E$ 与 $g^*E$ 丛同构.
    \end{proposition}

    在证明之前, 我们先分析一下命题. 设 $H:X\times[0,1]\rightarrow Y$ 是从 $f$ 到 $g$ 的光滑伦移, 即 $H|_{X\times\{0\}}=f$, $H|_{X\times\{1\}}=g$. 则有 $X\times[0,1]$ 上的拉回丛 $H^*E$, 
    且 $H^*E|_{X\times\{0\}}=f^*E,\,H^*E|_{X\times\{1\}}=g^*E$. 因此为了证明 $f^*E\cong g^*E$, 只需证明:

    \begin{proposition}[向量丛在柱空间的上下底的限制是同构的]
        当 $X$ 仿紧时, 对任意 $X\times[0,1]$ 上的向量丛 $E$, $E|_{X\times\{0\}}\cong E|_{X\times\{1\}}$.
    \end{proposition}
    \begin{proof}
        我们需要两个关于向量丛的事实:

        $(1):$ 若 $p:E\rightarrow X\times[a,b]$ 在 $X\times[a,c]$ 和 $X\times[c,b]$ 上分别是平凡的, 则 $E$ 在整个 $X\times[a,b]$ 上平凡. 只需分别写出在 $X\times[a,c]$ 和 $X\times[c,b]$ 
        上的平凡化 $h_1$ 和 $h_2$, 并修改 $h_2$ 使得它们在 $p^{-1}(X\times\{c\})$ 上匹配, 则 $h_1$ 和修改后的 $h_2$ 合并成整个 $X\times[a,b]$ 上的平凡化.

        $(2):$ 对于向量丛 $p:E\rightarrow X\times[0,1]$, 存在 $X$ 的开覆盖 $\{U_{\alpha}\}$ 使得 $E$ 在每个 $U_{\alpha}\times[0,1]$ 上都是平凡的. 对任意 $x\in X$, 存在 $U_{x,1},\dots,U_{x,k}$ 
        以及 $0=t_0<t_1<\cdots<t_k=1$ 使得 $E$ 在 $U_{x,i}\times[t_{i-1},t_i]$ 上平凡, 令 $U_x=U_{x,1}\cap\cdots\cap U_{x,k}$, 则由 $(1)$ 知 $E$ 在 $U_x\times[0,1]$ 上平凡.

        下面我们来证明该命题, 
    \end{proof}