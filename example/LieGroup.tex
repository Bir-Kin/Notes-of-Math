\chapter{\Lie 群}
    \section{\Lie 群的连通性和单连通性在重要定理中的作用}
        开始前我们先叙述 \Lie 群中的一个重要定理:
        \begin{theorem}[\Lie 代数同态提升为 \Lie 群同态]\label{thm:Lifting-LieAlgebraHom-to-LieGroupHom}
            设 $G,H$ 是 \Lie 群, $G$ 既连通又单连通, $\mathfrak{g},\mathfrak{h}$ 分别是 $G,H$ 的 \Lie 代数.
            若 $\rho:\mathfrak{g}\rightarrow\mathfrak{h}$ 是一个 \Lie 代数同态, 则存在唯一的 \Lie 群同态 $\Phi:G\rightarrow H$ 
            满足 $\md\Phi = \rho$.
        \end{theorem}
        需要注意定理中 $G$ 的单连通和连通的条件缺一不可.
        \begin{example}
            若 $G$ 不是单连通的, 则这样的 $\Phi$ 不一定存在.

            \Lie 群 $(S^1,\cdot)$ 和 $(\mR,+)$ 的 \Lie 代数均为 $\mR$, 但不存在 $S^1$ 到 $\mR$ 的非平凡 \Lie 群同态. 
            设 $\varphi:S^1\rightarrow\mR$ 为 \Lie 群同态, 取 $S^1$ 的一个稠密子群 $\me^{i\pi\mQ}:=\left\{\me^{i\pi\theta}\,\big|\,\theta\in\mQ\right\}$, 
            则因为 $\me^{i\pi\mQ}$ 中的元素都是有限阶的, $\varphi(\me^{i\pi\mQ}) = \{0\}$. 由 $\varphi$ 的连续性可得 $\varphi(S^1) = \{0\}$. 因此 $\varphi$ 只能是平凡群同态.
        \end{example}
        \begin{example}
            若 $G$ 不是连通的, 则就算每个连通分支都是单连通的也不一定存在这样的 $\Phi$.

            考虑 $\mR\rtimes\mZ_2$, $\mZ_2$ 在 $\mR$ 上的作用由 $0\rightarrow\id,\,1\rightarrow-\id$ 给出. $\mR\rtimes\mZ_2$ 和 $\mR$ 的 \Lie 代数均为 $\mR$, 
            但 $\mR$ 到自身的恒同映射无法提升为 $\mR\rtimes\mZ_2$ 到 $\mR$ 的同态.

            假设这样的同态 $\varphi$ 存在, 取 $\mR\rtimes\mZ_2$ 包含 $(0,0)$ 的分支, 它是连通且单连通的 \Lie 群, 因此由定理{\rm\ref{thm:Lifting-LieAlgebraHom-to-LieGroupHom}}的唯一性知
            \begin{align*}
                \varphi:\mR\times\{0\}&\rightarrow\mR \\
                (t,0)&\mapsto t,\quad\forall\,t\in\mR
            \end{align*}
            又因为 $(0,1)$ 是 $\mR\rtimes\mZ_2$ 的 $2$ 阶元, 因此 
            \begin{equation*}
                \varphi:(0,1)\mapsto 0
            \end{equation*}
            但是
            \begin{equation*}
                \varphi\Big((0,1)\cdot(t,0)\Big) = \varphi(-t,0) = -t \neq 0+t = \varphi(0,1)+\varphi(t,0),\quad t\neq0.
            \end{equation*}
            因此这样的群同态 $\varphi$ 不可能存在.
        \end{example}

    \section{非紧\Lie 群的非完全可约表示}
        \begin{example}
            我们知道紧李群的有限维表示是完全可约的, 但是如果李群非紧, 则很容易找到反例.
            
            比如考虑 $G=\mR$ 的二维实表示 $V = \mathrm{span}_{\mR}\{v_1,v_2\}$:
            \begin{align*}
                \rho:\mR&\rightarrow\GL(V) \\
                x&\mapsto\begin{pmatrix}
                    1 & x \\
                      & 1
                \end{pmatrix}
            \end{align*} 
            它有一个不可约表示 $V_1 = \mathrm{span}_{\mR}\{v_1\}$, 但是找不到 $V_1$ 的补表示.
            假设存在 $V_1$ 的 $G$-不变补空间 $V_2$, 取 $V_2$ 的基向量 $u = av_1+bv_2\:(b\neq0)$, 
            则 $\rho(1)(u) = (a+b)v_1+bv_2\in V_2$, 而 $u, \rho(1)(u)$ 线性无关, 这表明 $V_2$ 只能为整个 $V$, 矛盾!
        \end{example}

    \section{非紧连通\Lie 群不一定存在非平凡环面子群}
        \begin{example}
            对于紧连通李群 $G$ 而言, 任取其中的某个元素 $g$, 考虑 $g$ 生成的子群 $H$ 的闭包 $\mathrm{cl}H$ 的单位连通分支 $(\mathrm{cl}H)_0$, 
            它是紧连通 $\mathrm{Abel}$ 群, 故为 $G$ 的环面子群.
            
            若 $G$ 不紧, 则 $G$ 不一定存在非平凡环面子群, 比如 $G=\mR$. 
            一个不太平凡的例子是 $\mathrm{Heisenberg}$ 群 $H_3(\mR)$, 
            \begin{equation*}
                H_3(\mR) = \left\{\left.
                \begin{pmatrix}
                    1 & a & c \\
                      & 1 & b \\
                      &   & 1
                \end{pmatrix}\,\right|\,a,b,c\in\mR\right\}
            \end{equation*}
            它的李代数  为
            \begin{equation*}
                \mathfrak{h} = \left\{\left.
                \begin{pmatrix}
                    0 & a & c \\
                      & 0 & b \\
                      &   & 0
                \end{pmatrix}\,\right|\,a,b,c\in\mR\right\}
            \end{equation*}
            $\mathfrak{h}$ 有一组基
            \begin{equation*}
                X = \begin{pmatrix}
                    0 & 1 & 0 \\
                      & 0 & 0 \\
                      &   & 0
                \end{pmatrix},\quad
                Y = \begin{pmatrix}
                    0 & 0 & 0 \\
                      & 0 & 1 \\
                      &   & 0
                \end{pmatrix},\quad
                Z = \begin{pmatrix}
                    0 & 0 & 1 \\
                      & 0 & 0 \\
                      &   & 0
                \end{pmatrix}
            \end{equation*}
            它们之间满足
            \begin{equation*}
                [X,Y]=Z,\quad[X,Z]=0,\quad[Y,Z]=0
            \end{equation*}
            且指数映射
            \begin{align*}
                \exp:\mathfrak{h}&\rightarrow H_3(\mR) \\
                \begin{pmatrix}
                    0 & a & c \\
                      & 0 & b \\
                      &   & 0
                \end{pmatrix}&\mapsto
                \begin{pmatrix}
                    1 & a & c+\frac{ab}{2} \\
                      & 1 & b \\
                      &   & 1
                \end{pmatrix}
            \end{align*}
            是 $1-1$ 映射. 由此可知 $\mathfrak{h}$ 的任何一个一维子空间经指数映射后都不可能对应一个环面子群.
        \end{example}